Goal: use the survey to show that the theory outlined in the first chapter is applicable elsewhere.

0. Pivot to non-profit as a different domain
1. General theory so want a very different domain to test in
2. Similarities between non-profits and militant groups
    a. Similar mission-focused
    b. Similar logistical challenges
        1. Financing when (mostly) can't promise returns
        2. Fundraising environment changing

3. Differences
    1. Legal operating vs clandestine operating
    2. Internal oversight and legal recourse
    3. Differences in exit options
        a. Nominally, exit is easier, and it is certainly much easier to let go of unneeded staff
        b. [Put "myth of no-exit" here]
        
    
I. Survey
1. When fielded
2. Number of respondents
3. Who are respondents
    a. Where are they?
        1. Summary tables of state distribution
        2. Maps of IP address
    b. What do they do?
    
II. How do the components of the theory travel?

Mechanisms of bottom-up transformation, from the first chapter:
\item Recruit for reasons other than mission alignment
\item Recruit outside of socializing bandwidth 
\item Negotiation model of decision-making
\item Accommodation outcome


III. Can we induce accommodation through scarcity (conjoint)
1. Introduction of conjoint motivation: theory predicts that one major driver of rapid  expansion is acute budget shortfalls

2. As the surveys of nonprofit financial health indicate, the situation is likely to be salient to non-profit leaders

3. Some organizations- such as Human Rights Campaign-- are known for subsidizing local staffing. Thus the tradeoff may be one that the respondents have already considered.

4. Cons: Without any real trade-off, the experiment is susceptible to respondents reporting what they think they'd like to do, not what they might actually do. Thus, it is very suceptable to a desirability bias, and should be viewed as an extremely conservative estimate.

Secondly, the conjoint experiment was designed to test the pathway in which leaders are driven by scarcities to admit staff who will bring heterogeneous preferences. It does not explicitly test the mechanism which arose from the case studies, in which large influxes of new personnel arise after successes.


5. Conclusion and other expectations of portability
IV. Chapter Appendices:
    1. Survey
    2. Specific Question Coding Rules
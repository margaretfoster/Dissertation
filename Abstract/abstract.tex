\abstract
When do recruitment windfalls strengthen organizations while threatening their leader’s perception of success? This paper introduces a theory of grassroots-driven organizational change that is broadly applicable when leaders balance short-term survival with long-term mission focus. 
I introduce the concept of the \say{personnel resource curse} in which recruitment windfalls simultaneously strengthen an organization while undermining the leader’s ability to achieve their goals. I argue that upward- driving internal pressures caused by incomplete socialization of grassroots members can transform the priorities and operational focus of resource-constrained organizations. When this happens, leaders experience pressure to reorient their organization towards the preferences of the base, even if these preferences are not the same as the leader’s vision. The process and outcome are surprising, as the theory identifies contexts in which even strategic leaders will recruit cohorts that exceed their socializing capacity and who will subsequently initiate this change process. An undertheorized avenue of organizational change, grassroots-driven, and bottom-up transformational pressures can constrain group operations, produce internal stressors, and influence the trajectory of political and social movements.

The dissertation uses a multimethod approach to build a general theory of organizational transformation. I introduce the theory and frame the dissertation using case studies and a simple formal model of leader-recruit negotiation.  The heart of this theory is a negotiation-centric view of organizations, in which leaders require at least some degree of consent from the rank-and-file to adopt specific actions. This approach leads to a model of organizational decision making that is sensitive to changes in leverage and introduces avenues through which leaders can be induced to accommodate the preferences of members whose presence is critical to the organization’s effectiveness. The model of organizational transformation developed in this dissertation is applicable in a wide range of contexts, from militant groups struggling to operate and expand, to issue-based organizations that seek an influx of resources and skills, to decentralized political organizations that lack strong mechanisms of control and socialization. To demonstrate generality, this dissertation presents the results of a survey of United States-based non-profit leaders and managers, finding that experience with these dynamics is prevalent in the sample.
 
Understanding the impact of grassroots-driven and bottom-up transformational pressures on the evolution of organizations has a wide array of implications, from philosophical questions about how organizations maintain their identity and priorities to tactical conclusions about how to best nurture or combat organizations undergoing internal transformations. The research makes theoretical and empirical contributions to social scientific theories about organizational dynamics and the evolution of organizations.

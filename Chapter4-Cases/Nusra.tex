\chapter{Spotlight on Upward Pressure: Jabhat al-Nusra}
\section{Introduction}

This chapter highlights how the strategies that leaders have tried to use to manage bottom-up pressures for change.  It uses a focus on the presence and eventual widespread exit of foreign fighters in the Syrian Jabhat al-Nusra (also known as the Nusra Front) to highlight the tensions and stresses that heterogeneous preferences can bring to brigade commanders. 

In the narrative that follows, I argue that a vocal contingent of Jabhat al-Nusra's rank-and-file maintained an initial preference for a transnational jihadi revolutionary project rather than a Syria-specific resistance movement. This contingent articulated and acted on their preferences in ways that rendered it more difficult for Jabhat al-Nusra to maintain pragmatic and strategically-necessary relationships with other militant groups and the local populations. This chapter uses primary and secondary sources from alleged militants and commanders to illustrate the consequences of a lack of buy-in and focuses on the frustration and resentment described by recruits.

The section highlights the 2013-2015 conflict between Jabhat al-Nusra and the Islamic State of Iraq and al-Sham (ISIS or the Islamic State), as the competition for militant personnel revealed strategies for managing preferences of the rank and file. The close attention Jabhat al-Nusra provides an opportunity to highlight the strategies that leaders use to attempt to avoid or mitigate the pressure to accommodate the preferences of their base.  As well, the chapter highlights the limitations of these strategies, with Jabhat al-Nusra's ability to generate local alliances limited by their association with jihadism.

Jabhat al-Nusra initially appeared to cede momentum and personnel to ISIS but ultimately seems to have used the conflict to reduce their internal preference divergences.\footnote{Conversely, after gathering fighters seeking to not only fight local political leaders, but also looking for social revolution, ISIS appears to have tried several strategies to reduce internal preference divergence: restricting exit, systematically using some fighters in high-risk deployments, and imposing draconian internal socialization to shape the preferences and control the behavior of their soldiers. The ISIS strategy of imposing movement controls and using force to remove variation in internal preferences represented a gamble that they would retain, or even expand, their coercive capacity. Ultimately, the Islamic State's aggression triggered an international coalition to degrade its strength. As well, the brigade structure of ISIS highlights the tradeoff between decentralization to reduce the need for top-down management in a group under security pressure with the aggregation of heterogeneous preferences.  Although ISIS was very aggressive about limiting exit opportunities, which should give the leaders more ability to control the group, they also had a robust brigade culture, which seemed to serve as a hub for the political, social, and ideological variation among fighters~\autocite{mironova2019freedom, weiss2015isis, weiss2016isis}. Homophily in the foreign fighter brigades is a way in which the group never imposed uniform ideological preferences internally, with the result that various battalions were essentially fighting different wars. Thus, for example, Chechen brigades primarily motivated by fighting Russia could insist on being deployed on fronts where they would fight Russians in Syria.}

\subsection{Background}

Jabhat al-Nusra was established in Syria in August 2011 when Ahmed Hussein al-Shar' a, known as Abu Mohammed al-Jolani, was sent to Syria by the Iraq-based Islamic State.\footnote{The roots of both groups are deep and border-spanning:  both Jabhat al-Nusra, its successors, and the Islamic State of Iraq and al-Sham can trace their origin directly to insurgent and jihadi mobilization during the American invasion of Iraq in 2003. Their ideological underpinning has origins in the jihadi mobilization into Afghanistan during the 1980s.} Jolani, and six other men, were directed to capitalize on the nascent Syrian uprising to create the Syrian branch of the al-Qaeda organization~\autocite{abouzeid2014nextdoor}.  Within a year had rapidly gained prominence for effective fighting, reliable funding, and consistent supplies~\autocite{bbc2016jfs, ahad2013nusra}. The group eventually become one of the most active and influential rebel groups in the eastern battleground of Deir ez-Zour and Raqqah, before developing and maintaining a stronghold in Lattakia and Idlib.\footnote{Deir ez Zour is notable as a smuggling route for Syrian fighters into Iraq during the American occupation; the region also has the Conoco oil fields which provided significant operating resources to Jabhat al-Nusra, but which went to ISIS when the two groups split~\autocite{bauer2019behind}.}

An acrimonious split with the Islamic State in 2013 was followed by several years of intense fratricidal conflict both on the ground and within the wider jihadi ideological community. Jabhat al-Nusra subsequently carried out two attempts to rebrand their identities, into Jabhat Fatah al-Sham in 2016 and Hay' at Tahrir al-Sham in 2017.  Each name change was intended to bolster the Syrian credentials of the group, distance them from their al-Qaeda origins, and render the movement into a more appealing centerpiece of a coordinated Syrian resistance~\autocite{bbc2016jfs}.  However, the perception of Jabhat al-Nusra as prioritizing an Islamic revolution,  frustration with the hardline foreign forces lead to an early listing of the group as a terror organization. The perception of Jabhat al-Nusra as primarily seeking to establish an Islamic government created hesitation among other Syrian rebel groups to participate in conflict alongside the group and significantly restricted their battlefield and support options.

Jabhat al-Nusra's leadership has been described as comprised of Syrian and international al-Qaeda organizers and leaders and as deeply committed to a transnational revolutionary mission of jihadi statebuilding ~\autocite{lister2016profiling}. However, their strategy appeared to be one of an incremental social transformation via pragmatic alliances with other Syrian revolutionary groups. Although necessary, these alliances and coordination present a potential management challenge because it requires the type of realpolitik and compromises that the most ideologically-motivated fighters have, in the past, vociferously and bitterly rejected.\footnote{Beyond the Syria context that this chapter outlines, there a distinct genre of dissident foreign fighters vociferously complaining about their groups engaging in on-the-ground realpolitik. Occasionally, they describe their efforts to single-handedly undermine what they view as rejection of correct operations. See, for example~\cite{collins2002my, hammami2013FA1}.}

By 2015, Jabhat al-Nusra wanted to present itself as becoming embedded in local communities and working incrementally to develop support through local administration and social services. For example, speaking to VICE News, a Nusrsa Front commander described the \say{second generation} of the al-Qaeda revolution as using dispute adjudication and service provision to generate local support that would allow them to embed within the Sunni population~\autocite{dareih2015unmasked}. The strategy articulated in the interview with a reported Jabhat al-Nusra commander in Jabhat al-Nusra's northwest Syrian stronghold reinforced the strategic emphasis on collaboration with other rebel groups and local, context-specific, experiences and knowledge. He described Jabhat al-Nusra fighters as: \say{They fought the regime with other brigades, they lived among the people, they went up to the pulpits and gave the Friday sermon, they gave lessons and lectures, they have media activities, they fought in the frontline and gave martyrs and leaders.}~\autocite{dareih2015unmasked}.

%%
\subsection{Information Environment}

At the time of writing, the Syrian civil war has continued for nine years, at enormous cost to the Syrian population. The Syrian conflict is characterized by a huge influx of rebel groups and foreign participants, fighting on behalf of both the Syrian regime and the rebel movement as well as a nearly-unprecedented level of penetration by media and observers. Similarly, the accessibility of smartphones, satellite, and cross border mobile data networks has allowed for a tremendous amount of outflow of information from participation. 

The international interest in the civil war and access to telecommunications technology among the Sunni rebel forces means that there is an enormous amount of open-source information about the war.  Moreover, the factional conflict that played out between jihadi factions created an audience for members of the rank-and-file to articulate their grievances. The combination of interest and accessibility provides an opportunity to derive insight into the preferences of individuals at both the top and bottom levels of a militant group.  However, the data environment is neither random nor systematic. It prioritizes the perspectives of participants with communications access and a desire to share their preoccupations. In particular, the information environment is conducive to disproportionate weighting aggrieved international fighters, who are not only more motivated to share their frustrations but who are also aware of intense interest in their experiences.

Likewise, interviews with alleged commanders are similarly unrepresentative, but nevertheless, demonstrate an underlying consistency among the preferences at the top and middle ranks of Jabhat al-Nusra. Other analysts with apparent contacts maintain that the leadership of Jabhat al-Nusra has a longstanding desire to oversee the implementation of a jihadi state and religious-based governance~\autocite{icg2019best, lister2016profiling}. However, from the perspective of tension between radical troops and their leadership, the most important cleavage is one of pragmatism. As the International Crisis Group summarized, these preferences consistently relate a narrative of leaders motivated for a jihadi cause, but who have \say{repeatedly reached accommodations...that violate jihadist orthodoxy but, for the time being, ensure the group's survival}~\autocite{icg2019best}. 

Although the available primary sources about individual experiences are not necessarily broadly representative of sentiments within the group, these sources do articulate challenges faced by group leaders and are thus useful to illustrate the underlying negotiation process theorized by this manuscript.

\subsection{Internal Demographic Pressures}

The personnel landscape of the Syrian Civil War is an illustrative example for theory because the fluidity and aggression of the conflict amplify the perception and resources feedback loops that power the personnel resource curse described by this dissertation.  The fluidity between groups means that any group or brigade that begins to seem precarious risks entering a downward spiral of shedding manpower and resources. 

The fluidity of money, personnel, and supplies is a consistent theme in reporting of the on-the-ground dynamics from the Syrian conflict, particularly in the early years of the civil war.  The precariousness of manpower and resources, particularly in the early days of the conflict, were memorably described by the Iraqi journalist Gaith Abdul-Ahad, writing in the London Review of Books in February 2013. Chronicling a fissure within a  Free Syrian Army battalion defending Aleppo, he highlighted that commanders needed to both satisfy the physical and activity needs of their fighters. In the scene, the commander is trying to retain the soldiers for the \say{important defensive position} in Aleppo because, as he maintains, \say{if [the others] go, two frontline posts will be left empty...[regime forces] will be able to skirt around us.} Pleading for the organizer not to leave, he asked: \say{Did we do anything wrong? Didn't we feed them properly? Didn't they get their daily rations? Whatever ammunition we get we divide equally: tell me what we did wrong.}\autocite{abdul2013start}

The commander learns that the problem is twofold: the fighters are bored, and they have lined up an external patron. This patron, a Syrian based in the Gulf, will pay for ammunition if the new battalion sends videos of their operations. Not only will they be funded and able to pursue battles, but with their new sponsor, the departing fighters will no longer have to share any spoils that they capture~\autocite{abdul2013start}.

The unstable personnel resources that that any commander could expect to draw on resulted in a constant state of precariousness. Leaders tried to maintain their ranks and ensure a stable cadre of fighters that they could expect to draw on. The instability provided an initial appeal for jihadi fighters: within the first year of the conflict, they were notably better resourced and more cohesive than the fractious non-jihadi rebel movements.  Jihadi groups gained strength from the enhanced cohesion that is a benefit of ideological-commitment and the consistency of their funding. But the intense ideological commitment of some of the troops---and particularly for the foreign fighters --- created tensions with more pragmatic wings. These tensions---and the relatively consistent information from the groups brought about by the intense interest in, and scrutiny of, militant jihadi organizations--- create an opportunity to spotlight two different tactics in harnessing the benefits of more fighters while not being taken over by their preferences.  

At the outset of the conflict, Jabhat al-Nusra accepted foreign brigades. Foreign fighters often had useful training.  In Syria foreign fighters brought prior military experience-- including training from the North Atlantic Treaty Organization (NATO)--- and expertise in technical skills such as sniper training, military vehicles, topography, and weapons manufacture. They could, and did, act as trainers to diffuse these skills through their new rebel groups~\autocite[136]{mironova2019freedom}. Indeed, there is even a jihadi private military contractor, Malhama Tactical comprised of elite foreign fighters who present themselves as being trainers of elite Sunni rebel forces~\autocite{komar2017blackwater}.\footnote{Jihadi movements have benefited from the training that foreign fighters bring since the 1980s, when several early Afghan camps were staffed by trainers who had military experience in their home countries~\autocite{anas2019mountains, hamid2015arabs, levy2017exile, miller2015audacious, nasiri2007inside}.}  

Foreign troops were more willing to introduce aggressive tactics, such as suicide bombings. These tactics allowed Jabhat al-Nusra to maximize its battlefield effectiveness. Throughout rebel territories, as the conflict continued, foreign fighters were instrumental in effectuating the aggressive tactics that boosted the reputation of Jabhat al-Nusra as one of the most effective fighting units in the conflict. In particular, as in Holms, Jabhat al-Nusra was able to leverage vehicle-borne improvised explosive devices (VBIEDs) driven by suicide bombers to soften military bases for subsequent waves of ground troops. These suicide bombings and the bravery of their fighters in battle made Jabhat al-Nusra a desirable group for alliances and cooperation among the anti-regime forces, which then bolstered Jabhat al-Nusra's local profile and influence~\autocite{lister2016profiling}.  The first of these suicide bombers were reportedly foreign fighters, many from the conflicts in Iraq and Afghanistan, although quickly matched by recruits from within Syria~\autocite{ignatius2012affiliate, sherlock2012inside}. 

Foreign connections were also used by Jabhat al-Nusra to generate flows of recruitment and financing through existing global jihadi networks~\autocite{sherlock2012inside}. The money that foreign fighters brought with them was extremely appealing, given the intense competition between funders outside and inside of the country for influence on the policies of the groups that they underwrote and Jabhat al-Nusra's relative lack of a primary external sponsor~\autocite{heller2016keeping}. Although the lack of major external funders allowed Jabhat al-Nusra some distance from the demands of foreign state backers, it did increase the importance of local money-making ventures. In turn, these ventures relied on maintaining pragmatic and mutually-beneficial relationships with other conflict actors.\footnote{The group also sought out additional revenue streams, such as taxation and kidnapping journalists.} For example, writing in 2016,  analyst Sam Heller observed: \say{Ahrar al-Sham and Jabhat al-Nusra are excluded from the main conduits of international support and thus less susceptible to foreign pressure... al-Nusra has no state backers who can pull its strings, although that also means it has had to create its own resources}~\autocite{heller2016keeping}.

The group began to attract an increasing number of foreign fighters, and particularly those motivated by Jabhat al-Nusra's status as a Syrian branch of the Islamic State of Iraq.  Illustrating the internationalist recruitment, a self-proclaimed Jabhat al-Nusra commander in Holms issued a general invitation for Muslims to join the group's ranks: \say{Any pure Muslim can join Jabhat al-Nusra, but they have to be committed to Allah and fighting only for Allah.~\autocite{sherlock2012inside}. }To demonstrate religious motivation, the leader cited a common shibboleth used by jihadi fighters: willingness to cease smoking.\footnote{Not smoking was one way in which jihadis regarded themselves as morally and physically superior to non-jihadi fighters. In effect, abstaining from nicotine was a costly signal of  commitment~\autocite{hegghammer2013recruiter}.} The leader in Holms insisted that recruits should show their commitment by refraining from smoking, adding that: \say{If they are smokers and they die in Syria, how do we know that they died for God – and not because they were trying to go to reach a place to buy another packet?}~\autocite{sherlock2012inside}.

By the end of 2012, it was common for reporters and analysts to describe Jabhat al-Nusra as a largely-foreign force. For example, a November 2012 article from France24 described Jabhat al-Nusra as \say{made up of mostly foreign militants from Iraq, Saudi Arabia and Central Asia} who were willing to coordinate with local Syrian militias but were ultimately working for an Islamist project\autocite{fvc2012jabhat}. Among the motivations described by foreign fighters arriving in the first waves, from 2012-2013, were a desire to transfer frustration over local repression into a battleground where they could be active. This motivation is particularly closely associated with Chechen fighters, who quickly became significant military actors. Speaking to Joanna Paraszczuk, the commander of a private jihadi military contractor in Syria, reflected on his arrival in Syria in 2013 and summarized this perspective. Noting a longstanding frustration with the lack of opportunities for military action in Chechnya, he said:

\begin{quote}...in 2012-2013, news from Syria was widely disseminated. The situation in Syria was extremely clear, and reminiscent of the situation in Chechnya during the war. This really caught my interest, all the more so because at the time in Syria there were Chechens who were taking part in the fight against Assad. I had two choices: to endure and wait for a suitable moment to take action in my homeland, or go to Syria to help the Muslims against the bloody regime of Bashar al-Assad. I consulted with the brothers, and I decided to go to Syria. \autocite{paraszczuk2019ali}\end{quote}

In the first years of foreign fighter emigration to Syria, there were indications that Jabhat al-Nusra was intentionally focusing on recruiting from Arab communities so that their fighters would have sufficient language skills to be able to integrate with their command structure. 
Coordinators for Jabhat al-Nusra spoke to reporters about facilitating foreign volunteers and weapons from Lebanon into Syria. One facilitator who brought fighters through the countryside around the northern city of Holms bragged that: \say{I have sent in brothers from Saudi Arabia, Iraq, Pakistan, Lebanon, Turkmenistan, France and even from Britain} to join the Holms branch of Jabhat al-Nusra or to link up with other groups within Jabhat al-Nusra to conduct combined operations~\autocite{sherlock2012inside}. 

During the influx of foreign fighters, Jabhat al-Nusra began to take on a transnational cast. The internationalization can even be seen in the ways in which local commanders began to describe the organization, framing the Syrian civil war as a transnational religious conflict.  For example, the leader of Jabhat al-Nusra brigade in Holms described the Syrian conflict using the same religiously-inflected language and framing, calling the group \say{the sword of the Islamic land}~\autocite{sherlock2012inside}. The description is notable for, firstly, using a metaphor that speaks to a historical and transnational Islamist literature tradition and, secondly, for describing territoriality based on religion rather than a Syrian political identity. The enhanced profile and reputation for military success was valuable for a group that was aspiring to become central to the Sunni resistance~\autocite{lister2016profiling, bbc2016jfs}.
%%
\subsection{Downsides of Foreign Fighters}

Despite their utility, foreign fighters came with a cost. These new fighters jeopardized the Syrian identity of Jabhat al-Nusra\autocite{adraoui2019case}.\footnote{The growing reputation of Jabhat al-Nusra as a foreign-driven extremist group aided the Syrian regime. It was thus intentionally amplified in 2011 when the government released dozens of jihadi prisoners and invested significant effort in portraying the entire anti-Assad uprising as an extremist plot to establish a transnational jihadi state in the Levant. This strategy has been extensively covered by media sources, including \autocite{salloum2013jail, weiss2016isis}.}  Moreover, many arrived in Syria with the expectation that their desired Islamic revolution and aggressive implementation of hardline religious laws would take strategic priority.  The ensuing insubordination led to friction between fighters and the local communities as well as to internal tension within Jabhat al-Nusra as the more extreme rank-and-file insisted that the group adopt their preferred strategies and tactics.

Summarizing the negative view of Arab foreign fighters, a Syrian doctor complained about this group being particularly harsh towards the local community.  He began by praising non-Arab foreigners for \say{being nice to the people} and investing in good-faith gestures, such as trying to communicate despite imperfect Arabic.\footnote{As the following memoirs might suggest, not all non-Arab fighters were respectful.} In contrast, the doctor associated Arab foreign fighters with a disrespectful and harsh relationship to the local community, adding: 

\begin{quote}
Tunisians, Yemenis, and Saudis, on the other hand, were so mean and harsh on people. They used to enforce the rules by [threatening with] weapons, and if you didn't listen, they would arrest you. They didn't discuss anything, while the non-Arabs often initiated discussions.\end{quote}\footnote{Although the doctor's recollections were primarily about ISIS foreign fighters, and Jabhat al-Nusra subsequently went to great lengths to limit potential friction points between their remaining foreign fighters and the local community, this recollection could have equally applied to the period before the split.}~\autocite[159]{mironova2019freedom}  

Primary sources material from Western foreign fighters are likewise quite transparent that the emigrants were often extremely abrasive to the local community. One memoir, attributed to British jihadi and propagandist Omar Hussain,\footnote{Hussain maintained an active online presence under the name \say{Abu Sa'eed al-Britani} before his apparent death in an Islamic State prison camp in Raqqah~\autocite{farrall2017abusaeed}. } described the poor relationship between foreign fighters and the local community, and in doing so, underscored the difficulties of managing an influx of recruits whose goals did not align with those of their group.  

Hussain's memoir, dated to mid-2015 and released in five parts under the title \say{Exposing Jabhat An Nusrah,} outlined his reasons for defecting from Jabhat al-Nusra to the Islamic State. He described at great length his irritation that Jabhat al-Nusra was unwilling to impose an extremist social policy on the Syrian population, which he felt that his participation in the conflict as an emigrant fighter empowered him to do.  Thus the document seethes with Hussain's frustration that, despite being armed, he was being restricted from imposing hardline Islamist social policies:

\begin{quote}
When I first arrived in Sham and walked the streets of Atmah, Dana, and its surrounding areas, I felt an immediate kick of excitement that I was in the land of jihad and carrying a gun gave me a sense of honor. However, as the weeks went by, the initial excitement faded away, I started realising that there was a lot of vice and munkar [forbidden behaviors] around me..I started feeling as if I was back in Edgeware Road in London. The only difference being, that I had a gun… You couldn't even tell someone to stop smoking even though we had a gun. Hardly anyone in the town had respect for Jabhat An-Nusrah and their approach to implementing the Shariah was weak and futile~\autocite{hussein2015exposing}.\end{quote}

The diminishing of a Syrian identity for the movement is evident in one of the first interviews between a Western journalist and a member of Jabhat al-Nusra. The interview,  published online in March 2013, was  prefaced with a discussion of the difficulty of obtaining interviews from Jabhat al-Nusra members because the group was largely foreign. However, the preface noted that access had been easing \say{as more Syrians join the group and they make more gains on the ground in the fight against the Syrian government} ~\autocite{marrouch2013nusra}. Nevertheless, despite the statement that Syrian participation in the group was increasing, the identity of their interviewee, who claimed to be a 21-year-old from Libya, emphasized the international reputation of early Jabhat al-Nusra.  As well, in describing his \say{vision for Syria} he using framing and vocabulary that made no specific reference to Syria itself:

\begin{quote}
We Muslims have a certainty, there is talk of the prophet who preached that the best place on earth is the Levant. He also said that God chooses whom he wishes to be his best followers to be from here. I hope God chooses me and this is why I came to here, too ~\autocite{marrouch2013nusra}.\end{quote} 
Another 2013 interview with a Jabhat al-Nusra fighter, conducted by The Economist magazine, also featured transnational framing and a dismissive view of the preferences of locals and even of the national identity of Syria. The purported fighter, a Syrian, responded to a question about the future that he envisioned for Syria with a transnational vision, stating:

\begin{quote}
We want the future that Islam commands. Not a country with borders but an umma [worldwide Islamic community of believers] of all the Muslim people. All Muslims should be united~\autocite{pom2013nusra}.\end{quote}

When asked how the local population would support the project if they generally do not share the views of the Front, the fighter responded with indifference, stating:

\begin{quote}It would be great if the Syrians were with us but the kuffar [non-Muslims] are not important... The number with us doesn't matter.\end{quote}~\autocite{pom2013nusra}

These tensions are further evident in reports from dissident fighters who left Jabhat al-Nusra for ISIS, in part due to the defector's preferences to fight for an Islamist social revolution.  The British memoirist, Omar Hussain ~/ \say{Abu Sa'eed al-Britani} framed his jihadi memoir under the theme of \say{why I left} Jabhat al-Nusra, which he then circulated to social media outlets in 2015. 

Central to his sense of disillusionment with the group were his complaints about Jabhat al-Nusra prioritizing their own military goals instead of aiding the other Islamist fighters, and apparent lack of interest in pursuing Islamic governance, and their \say{fear of being labeled extremists.}  In the second of his messages, he describes frustration with Jabhat al-Nusra's unwillingness to simultaneously launch military and social campaigns: \say{they said that Dawah [outreach] was needed to be given to the people before we could judge them according to the Shariah, yet they were not so keen to give Dawah...So the fruits of Jabhat's jihad and struggle was nothing more than removing the tyrant}~\autocite{hussein2015exposing}

The aggregated influence of jihadi and foreign fighters, such as those quoted above, began to cost Jabhat al-Nusra. Heavily recruiting foreign jihadis enhanced Jabhat al-Nusra's military cohesion and battlefield efficacy in the critical early months of the civil war. The strategy nearly allowed them to take the mantle of the dominant Sunni group. However, an unintended consequence may have been to shift the demographics of the foreign fighters towards a pool of recruits inclined towards a coercive relationship with the local community. This had a devastating effect in two ways: first, their harshness alienated the Syrian community that Jabhat al-Nusra would rely upon as the fighting continued. Secondly, and more importantly, it cemented a perception of Jabhat al-Nusra as a jihadi insurgency, thereby chasing away potential allies, funders, and supporters unwilling to be associated with jihadism.


%%% 
\subsection{Attempts to Dictate Strategies and Tactics}

In addition to feeling entitled to impose hardline Islamic social policies on the local community, Jabhat al-Nura's jihadi recruits also attempted to dictate strategic and tactical decisions according to their expectations of Islamic doctrine.  Militarily questionable at best, this push became increasingly untenable as Jabhat al-Nusra sought to expand their recruits from other demographics. According to interviews with fighters at the time, foreign members of the group pushed to violate the leadership's polices against striking non-Muslim targets. One fighter reported internal tensions by noting that while native Syrian members were willing to follow al-Qaeda's stated policy of avoiding sectarian violence, \say{some of the foreign fighters hate the west and all non-Muslims… They want to attack churches. Personally, I don't like this. But this is how they were taught in Iraq and Chechnya.}~\autocite{sherlock2012inside}.

Another example of foreign fighters attempting to dictate the behavior of Jabhat al-Nusra appears to have undermined the battlefield realpolitik that the group intended to engage in to increase their material resources and the forces that they could command. In particular, one of Jabhat al-Nusra's moneymaking schemes was reportedly to facilitate and tax weapons transmissions between other organizations. Yet, this pragmatism was a liability to their extreme fighters, who subsequently lambasted the group for making deals with less ideological groups, not automatically supporting the other jihadi factions, and even facilitating attacks on Islamic State positions~\autocite{hussein2015exposing, sudani2015days}. 

Faced with instances of Jabhat al-Nusra not acting against secular or less-extreme Islamist brigades, and even facilitating weapons transfers that would be used against fellow jihadis, foreign fighters castigated what they viewed as cowardice and abdication of natural jihadi solidarity between the Islamist State and the Nusra Front. 

For example, the British fighter Omar Hussain/ Abu Sa'eed al-Britani reported being assigned to a group that guarded an arms deal between Jabhat al-Nura and the Free Syrian Army in March 2014. In his memoirs, he describes not only his disquiet at witnessing cooperation between the two rebel groups, but also his willingness to absolve himself of responsibility for following instructions when they conflicted with his understanding of religious permissions. He wrote that after the deal:

\begin{quote} Once in the safe-house, I was told not to tell anyone what I saw so the next few days went by with me telling everyone I knew about what I saw. The haqq [truth] needs to be told so why would I hide it? … If only I knew then what I did now, I would have opened fire upon them... I could have prevented FSA from gaining weapons and also I had such an easy opportunity to wipe many of them out.~\autocite{hussein2015exposing}.\end{quote}

Another defector from Jabhat al-Nusra to the Islamic State, writing under the name \say{Abu Bakr al- Jazrawi} shared a similar sense of frustration with Jabhat al-Nusra alliances. In a \say{testimony} included in Omar Hussain's \say{Exposing Jabhat An-Nusra} document, al-Jazrawi cited examples of agreements between Jabhat al-Nusra and a brigade under the command of Jamal Marouf, a powerful rebel leader who operated under the umbrella of the Free Syrian Army.  Al-Jazrawi described his rage in learning that Jabhat al-Nura had operational agreements with other fighters who were intending to attack the Islamic State. He related several instances of Marouf's troops traversing Nusra-held territory and retrieving weapons and ammunition stored with Jabhat al-Nusra. He and his comrades were \say{upset} at the repeated instructions from their leaders to allow the movement and supplies transferrals, concluding, \say{And this is the reason for me leaving Jabhat al- Nusrah}~\autocite{jazrawt2015exposing}.

\subsection{Syrianization Campaign}

The internal tensions and external consequences of their jihadi association began to limit Jabhat al-Nusra's effectiveness in the conflict, the leadership of Jabhat al-Nusra attempted a countervailing campaign of re-Syrianization. This strategic effort was aided by a falling out between Jabhat al-Nusra and the Islamic State.
  
As their profile rose in the conflict, the transnational origins of the Nura Front were revealed when al-Qaeda in Iraq  (AQI) announced that the leadership of Jabhat al-Nusra had been sent into Syria to establish a branch of the transnational jihadi insurgency in the country. 
In the announcement, the leader of al-Qaeda in Iraq, Abu Bakr al-Baghdadi, ordered Jabhat al-Nusra and Jolani to reintegrate with the Iraq-based jihadi group. 
Jolani refused the order, announced Jabhat al-Nusra's independence as a Syrian militant group, and announced that the group would answer only to al-Qaeda's central leadership. This lead to several years of bitter conflict between al-Qaeda in Iraq's newly regionalized Islamic State of Iraq and al-Sham (ISIS or ISIL) over which group was the legitimate physical and spiritual leader of the jihadi resistance to Bashar al-Assad.  In particular, ISIS leaned heavily into an aggressively transnational mission of social and political revolution, which they used to try to detach Jabhat al-Nusra's foreign cadres. 

Fighters and brigades rapidly switched affiliation, often articulating their preferences for the goals and tactics of one group over the other.  In dissident messages, Jabhat al-Nusra appears to have had considerable internal tensions over whether, and how, to be a Syrian anti-Assad movement or a jihadi group based Syria.  Foreign fighters defected from Jabhat al-Nusra to the Islamic State because they were more attracted to the transnational goals of the Islamic State. One common sentiment among fighters who described their reasons for switching from Jabhat al-Nusra to the Islamic State was that \say{Jabhat al-Nusra was only interested in fighting Assad while ISIS was dedicated to building an Islamic state and enforcing sharia law}~\autocite[127]{mironova2019freedom}.

As Jabhat al-Nusra began seeking independence and distancing itself from both the al-Qaeda and Islamic State transnational movements, their leaders encouraged a Syrian-first perspective in the movement. The resulting influx of Jabhat al-Nusra's foreign fighters into ISIS operated in similar ways as the fighters originally aided Jabhat al-Nusra, by providing shock troops and an aura of invincibility and momentum.  Claiming that 80\% of the foreign fighters had joined ISIS during the internecine conflict between the two groups, a Jabhat al-Nusra commander described the effects in devastating terms: the overwhelming majority of foreign fighters had defected to IS, while those remaining struggled with low morale. He said: \say{This has broken our spine. Many of our fighters became lax. They ask: \say{Why are we fighting if there is a dispute among the emirs?}} ~\autocite{ahad2013nusra}

The rapid defection of foreign fighters lead many within and outside of the transnational jihadi community to believe that the momentum of the conflict had shifted towards ISIS and away from al-Qaeda. However, the apparent gains made by the Islamic State may have concealed a deeper strategy by Jabhat al-Nusra to instrumentally change the demographics of their recruits and recast themselves as a native Syrian movement. In a retrospective interview with the International Crisis Group, Jolani claimed that the break with ISIS represented an intentional ideological shift to become more Syrian and less transnational. Stating that he allied the group with al-Qaeda because \say{we didn't have any good options}, Jolani claimed that he had always conditioned the alliance on his group remaining based in Syria and not using the country to stage external operations~\autocite{icg2020conversation}. In the years after the break and eventual renaming, the group \say{sidelined or expelled most hardline and non-Syrian voices in HTS who opposed its apparent ideological transformation, thus rendering it more Syrian and less transnational jihadist in orientation.}~\autocite{icg2020conversation}.

The interview alluded to Jabhat al-Nusra [under their most recent name of Hay'at Tahrir al-Sham] use of exit as a tool to effectuate this change, Jolani claimed: \say{With regard to what you describe as hardline voices within HTS, we have shown time and again that whenever we reach a decision about something, everyone follows the chain of command. As for those who don't, they can easily part ways with us.}

By late 2015, the conflict continued to internationalize with Russian intervention to support the Assad regime. Relationships between rebel groups became increasingly critical for survival, adding further importance to Jabhat al-Nusra's strategy of trying to coordinate and lead coalitions of Sunni rebel brigades, \say{fostering a constructive relationship of dependence…[and] pro-al-Qaeda support groups} among the armed opposition~\autocite[33]{lister2016profiling}. Moreover, no Sunni rebel group was strong enough to control the liberated territory and consolidate the myriad funding and logistics routes, least of all Jabhat al-Nusra who lacked a single state sponsor and who therefore tended to source their weapons and ammunition from booty captured during battle and from taxes imposed on allowing other groups' supply routes to operate~\autocite[32]{lister2016profiling}

 By 2016, Jabhat al-Nusra was continuing to position itself as one among the nationalist rebel movements. They began heavily recruiting from among the opposition communities in Aleppo and Idlib, leveraging perceptions that the international community had abandoned the Syrian people and cause. The local recruiting strategy paid off dramatically, with reports that Jabhat al-Nusra gained at least 3,000 Syrian forces in the first half of 2016~\autocite[6]{lister2016profiling}. In 2016, Jabhat al-Nusra tried again to unify the Sunni Syrian opposition, with Jabhat al-Nusra as the leader. This effort resoundingly failed, with prospective subsidiaries balking at Jabhat al-Nusra's connections to al-qaeda~\autocite{lister2018loss}.

On July 28, 2016 Jabhat al-Nusra rebranded themselves as Jabhat Fateh al-Sham (\say{Syria Conquest Front} or JFS ). Amid the rebranding announcement, JFS positioned itself as a local movement, with \say{no affiliation to any external entity} and described their intention to coordinate more closely with other opposition rebel groups in Syria~\autocite{jabhatfs2016announce, jabhatfs2016break}. The rebranding was widely believed to be a strategy to increase their grassroots support and was accompanied by relaxing their typically  \say{long and strict recruitment procedures} in order to add more flexibility and attract more local fighters~\autocite{haid2017behind}.

Interestingly, and suggesting that Jabhat Fateh al-Sham was still conscious of the influence of their foreign audience, on the same day as the announcement of their rebranding as Jabhat Fateh al-Sham, and among persistent rumors that the group's leadership was meeting to decide whether or not to break away from al-Qaeda's transnational leadership, Jabhat al-Nusra/Jabah Fateh al-Sham released a second video in which a speaker identified as the \say{deputy of [al-Qaeda leader] Ayman al-Zawahiri} was featured telling the Syrian branch to take any course of action needed to safeguard the Syrian jihad. In the video the alleged deputy counselled:

\begin{quote}
... we direct the leadership of Jabhat al-Nusra to go ahead with what preserves the good of Islam and the Muslims, and protects the jihad of the Syrian people. We urge them to take the appropriate steps towards this matter. This is a step from us, and a call from us to all the factions in Sham to unify in what Allah approves of, and to work together… Be as one rank that protects our people, defends our lands, and please our eyes with your unity by meeting on a wise Islamic government that restores truth and spreads justice among the Muslims. Allah permitting, we will be the first to support it~\autocite{jabhatfs2016deputy}.\end{quote}

However, despite these efforts, the former Jabhat al-Nusra was never able to shed their jihadi associations, nor, evidently, their jihadi tendencies. Not only did areas under their control periodically reject the imposition of Nusra governance, but during lulls the conflict with Regime forces, civilians also rejected Nusra aggression against Free Syrian Army factions without a reputation for corruption. One such example can be see in the defense of Division 13 launched by residents of Maraat al-Nu’man in March 2016~\autocite{cambanis2016syrian}. Moreover, as late as 2019, Hay'at Tahrir al-Sham spokesmen were seeking to bolster their manpower from among members of the Syrian diaspora, thereby tacitly acknowledging that they had failed to forge a unified Sunni opposition rebel movement and thus needed to look abroad. 

The framing of the recruitment appeals forms a striking contrast to the internationalist depiction of Jabhat al-Nusra's objectives from before 2016. These later calls specifically focus on members of the Syrian diaspora to return is a notable difference from the aggressive international recruitment platform of the early years of the conflict. For example, in October 2019, the group released a video message, titled \say{Strive With Your Life: Video Nasheed,} that called on members of the Syrian diaspora to return to Syria and join the fight.  As the following excerpt highlights, the nationalistic framing of the video is in striking contrast to the framing of previous messages from Jabhat al-Nusra recruiters.  Instead of presenting the conflict in transnational or religious terms and framing their appeals to a broad audience, the October 2019 message tries to make a connection with the specific and personal connections of native Syrians with their homeland. 

Where recruitment messages targeting foreign fighters often highlighted the purported openness and accessibility of Syria and the theoretically-transnational Muslim community, the 2019 message goes in the opposite direction by stating that Syria is a battleground to which members of the diaspora are already a part by virtue of their nationality:

\begin{quote} 
This is a message that I am directing to our beloved young people and our heroes. Our heroes who have left the fronts, and they stayed in the camps in cities and towns. Until when you avoid the land of glory, honor, dignity, and faith. These lands of heroism and knights, and you are its people. Until when you are away from these lands. These lands that whoever is away from them will have regret. Until when our heroes, why are you away from these sites?

Until when young people will you fill the camps with those holding back and sitting and women. These are the places of women, and here are the lands of men and of glory and honor. These are the lands that Allah the Almighty loves and is satisfied with its people, and He honors them with rewards and giving~\autocite{hts2019join}.\end{quote}

In the same year, Abdullah al-Muhaysini, a Saudi cleric based in Syria called on Syrians in Europe to return to the country to fight. Muhaysini framed the motivation of foreign recruits as having journeyed to Syria to \say{be behind} Syrian revolutionaries, \say{defend our land which is your land} rather than to \say{rule or to take any position here}~\autocite{muhaysini2019capital}. Directing listeners to reach out on social media for instructions on how to come back, Muhaysini framed incentive to return as one of nationalism and homeland:

\begin{quote}
“you will have remorse, and you will say I was in my country why did I leave it? Is this the reward for my country, is this the reward for my family, is this the reward of my imprisoned brothers?~\autocite{muhaysini2019capital}\end{quote}

However, the strongest indication of that HTS had failed to shed their jihadi association is featured in the International Crisis Group interview. Given at a time in which the Syrian regime and it's allies have retaken nearly all of the former Jabhat al-Nusra~/HTS stronghold of Idlib, Jolani tried repeatedly to distance himself from al-Qaeda and transnational jihad. For example, he downplayed his own jihadi motivations and background, saying of the group's origins and--- despite the activity and importance of al-Qaeda linked trainers and leaders during Jabhat al-Nusra's consolidation---presented the alliance with al-Qaeda as a decision taken reluctantly:

\begin{quote}
I was influenced by a Salafi-jihadist milieu that emerged from a desire to resist the U.S. occupation of Iraq...but today the reality on the ground is our reference [...] When we broke off from ISIS, we didn't have any good options. I had to take a quick decision, so I gathered my inner circle and told them I was considering pledging allegiance to al-Qaeda. They advised against it – some even described it as suicidal – but no one was able to provide me with an alternative~\autocite{icg2020conversation}\end{quote}



%% Probably going to need to use this as a future expansion
%Decentralization and internal heterogeneity: ISIS

%%with the Islamic State’s socializing ability hampered by the operational centrality of brigades and strong within-brigade homophily and
 
%Although ISIS was very aggressive about limiting exit opportunities, which should give the leaders more ability to control the group, they also had a very strong brigade culture, which seemed to serve as a hub for the political, social, and ideological variation among fighters [eg: Mironova 2019, 130-133 on homophilly in the brigades. Homophilly in the foreign fighter brigades is a way in which the group never imposed uniform ideological preferences internally, with the result that different brigades were virtually even fighting different conflicts. Thus, for example, Chechen brigades primarily motivated by fighting russia could insist on being deployed on fronts where they would fight Russians in syria [mironova ].

%"Professional fighters who had gone to Syria left when another conflict they were interested in started elsewhere. For example, one Chechen mercenary who had trained opposition groups in Syria at the beginning of the conflict left for the Ukraine when Russia invaded in 2014 and started his own battalion to train Ukrainian forces...Fighters of other nationalities, particularly ones from Central Asia went to Afghanistan when it became clear that ISIS in Syria and Iraq would fall.” [Mironova 2018, pg 135]

%%ISIS: pockets of transformation by foreign forces who were focused on a single external target, claims that ISIS allowed their foreign troops to encourage a transnational focus that would not benefit the local conflict (though, with ISIS need to be careful about the reports of complicity with the state while ISIS tried to consolidate the jihadi insurgency around themselves and while the state tried to undermine other groups.)

\section{Conclusion}

The previous chapter has used the example of extremist foreign fighters within Jabhat al-Nusra in Syria to highlighted ways in which the preferences of recruited fighters can exert pressure on leaders to shape a conflict in the way that the recruits prefer.
 
Jabhat al-Nusra’s apparent strategy of using the conflict with the Islamic State to purge their extremist revolutionary forces, entailed a twofold gamble. The first major risk was that the outflow of resources and personnel would not create the impression of a failing groups, while the second was that the other Sunni groups in Syria with whom they wanted to coordinate would accept their rebranding away from a transnational jihadi movement. Yet, Jabhat al-Nusra was unable to overcome their association with al-Qaeda enough to entice other rebel factions to unite with them. 




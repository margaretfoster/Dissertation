\chapter{Management Strategies and Consequences}
\label{chapter :cases}

%Within conflict variation: Syria

Syrian groups of interest:
%Free Syrian Army
%Jabhat al-Nusra/ Jabhat Fatah al-Sham (2016)/ HTS (2017)
%Ahrar al-Sham
%ISIS

\section{Introduction}

This chapter discusses downstream consequences of changes in rank-and-file makeup for the operation of organizations. It highlights as well the strategies that leaders have tried to use to manage bottom-up pressures for change. This can be particularly seen in the attempts of the Syrian Nusra Front to shed their transnational linkages and reputation as a rebel group made up of extremist foreign forces. The perception significantly restricted their battlefield and support options, leading to an early listing of the Nusra Front as a terror organization and hesitation among other Syrian rebel groups to participate in conflict alongside the group. Both of these jeopardized their early successes. As a result, there is notable divergence in strategy towards recruitment of foreign fighters between the Islamic State and the Nusra Front, as the former attempted to harness an upswelling of enthusiasm for adventure and revolutionary activism. This gave the Islamic State of Iraq and al-Sham (ISIS) an initial aura of success, leading many within and outside of the transnational jihadi community to believe that the momentum of the conflict had shifted towards ISIS and away from al-Qaeda. However, I argue in the following section, the apparent gains made by the Islamic State may have concealed a deeper strategy by the Nusra Front to instrumentally change the demographics of their recruits and recast themselves as a native Syrian movement. Conversely, the rapid influx of revolutionary foreign fighters into the Islamic State seeded internal preference heterogeneity that was difficult for the movement to overcome.  

The close focus on a few movements provide an opportunity to highlight the strategies that leaders use to attempt to avoid or mitigate the pressure to accommodate the preferences of their base. The three Syrian groups cover a range of possible strategies ranging from an apparent lack of strategy from the Free Syrian Army; to the Islamic State’s attempts to impose more and more internal control on their personnel to both increase the group’s socialization ability and restrict fighter’s options for exit; and finally to the Nusra Front’s top-down re-Syrianization. As well, the chapter highlights the limitations of these strategies, with the Islamic State’s socializing ability hampered by the operational centrality of brigades and strong within-brigade homophily and the Nusra Front finding that perceptions and reputation are difficult to change. 

The conflict in Syria is long and complicated. At the time of writing, the Syrian civil war has been going on for nine years, at enormous cost to the Syrian population. Moreover, the roots of both groups are deep and border-spanning:  both the Nusra Front and it’s successors as well as the Islamic State of Iraq and al-Sham can trace their origin directly to insurgent and jihadi mobilization during the American invasion of Iraq in 2003, and their ideological underpinings to jihadi mobilization into Afghanistan during the 1980s. 

 The conflict has also been  characterized by a huge influx of rebel groups and foreign participants, on both the rebel and state side as well as nearly-unprecedented level of penetration by media and observers. Similarly, the accessibility of smartphones, satellite, and cross border mobile data networks  has allowed for a tremendous amount of outflow of information from participation. The complexity of the conflict, and the fact that conflict remains ongoing makes it a difficult conflict to accurately summarize in totality. Instead, this chapter will open with an overview of the conflict, and then, drawing on both primary and secondary sources, will focus on a few moments of the conflict that highlight the impact of personnel-driven recruitment tensions.  

Rather than attempt to summarize an enormous and complex conflict, with an unprecedented amount of real-time information, the chapter highlights a few specific periods during the conflict: the extremist takeover of the Free Syrian army in 2011-2013 and the, the 2013-2015 conflict between the Nusra Front and ISIS. It uses these windows to highlight what the two Salafi groups’ competition for fighters and what the competition for militant personnel tells us about management of preferences of the rank and file. In particular, there is interesting variation in the management strategies of the two groups with the Nusra Front initially appearing to cede momentum and personnel to ISIS but ultimately using the conflict to reduce their internal preference divergences. Conversely, after gathering fighters seeking to not only fight local political leaders, but also looking for social revolution, ISIS appears to have tried a number of tools to reduce internal preference divergence: restricting exit, systematically using some fighters in high risk deployments,  and imposing a draconian internal socialization to shape the preferences and control the behavior of their fighters. At the same time, the bridge structure of isis highlights the tradeoff between decentralization to reduce the need for top-down management in a group under security pressure with the aggregation of heterogenous preferences. 


\section{Shared Context}

\section{Transformation: extremist takeover of FSA}

Used exit to avoid transformation: Nusra Front remaining Syria-focused. [allowed internationalist extremist to purge themselves, which made them look like they were losing the race for the jihadi position when ISIS was rising, but allowed them to keep a focus on the Regime]
ISIS: pockets of transformation by foreign forces who were focused on a single external target, claims that ISIS allowed their foreign troops to encourage a transnational focus that would not benefit the local conflict (though, with ISIS need to be careful about the reports of complicity with the state while ISIS tried to consolidate the jihadi insurgency around themselves and while the state tried to undermine other groups.)


Syrian cases share:
Urgency for recruitment: active conflict with aggressive repression by Syrian regime as well as cross-group predation. Huge amounts of fluidity between groups means that any group or brigade on the downswing is likely to end up out all of their people and materials (for example; the LRB essay about how to run a Syrian brigade with the brigade leader increasingly desperate for people.

\section{Jabhat al-Nusra’s Syrianization Process}

%[[The story that I want to tell is that at the outset of the conflict, the Nusra Front apparently accepted foreign brigades. 
Foreign fighters often had useful training-- a feature that has benefited jihadi movements since the 1980s, when some of the early camps were staffed by trainers who had military experiene in their home countries. This continued in Syria, where foreign fighters could bring military experience-- including North Atlantic Treaty Organization (NATO) training and expertise in technical skills including sniper training, military vehicles, topography, and weapons manufacture-- that they could diffuse through the rebel operations as  trainers~\autocite[136]{mironova}. Indeed, there is even a jihadi private military contractor, Malhama Tactical comprised of elite foreign fighters who present themselves as being trainers of elite rebel forces~\autocite{komar2017blackwater}. 

The Nusra Front was established in Syria in August 2011 [[story about the entry of Jolani and a few other]] and within a year had rapidly gained prominence for effective fighting, reliable funding, and consistent supplies~\autocite{bbc2016jfs}.The group eventually become one of the most active and influential rebel groups in the eastern battleground of Deir ez-Zour and Raqqah, before developing and maintaining a stronghold in Latakia and Idlib.\footnote{deir ez zour is notable as a smuggling route for Syrian fighters into iraq during the american occupation; the region also has the Conoco oil fields which provided significant operating resources to the Nursa Front, but which went to ISIS when the two groups split, https://www.motherjones.com/politics/2019/06/behind-the-lines-syria-part-one/}

%%[[Tell some of the Lister story about the top leadership of the Nusra Front being very dedicated to al-Qaeda’s transnational revolutionary mission, and that the group has had a few influxes of top-top al-Qaeda organizers and leaders. But that the strategy appears to be one of incremental social revolution and remaining Syrian enough to coordinate with other Syrian revolutionary groups. Note for example that this coordination is a potential management challenge because it requires the type of realpolitik and compromise that foreign fighters have, in the past, vociferously and bitterly rejected. Examples: Auki Collins, Shabaab, Abu Saeed Britani. 

In 2015, the group wanted to present itself as becoming embedded in local communities and working incrementally to develop support through local administration and social services. For example, speaking to VICE News, a Nusrsa Front commander described the “second generation” of the al-Qaeda revolution as using dispute adjudication and service provision to generate local support that would allow them to embed within the Sunni population~\autocite{dareih2015unmasked}. The strategy articulated in the interview with a reported Nusra Front commander in the Nusra Front’s northwest Syrian stronghold, reinforced the strategic emphasis on collaboration with other rebel groups and local, context-specific, experiences and knowledge. He described Nusra Front fighters as: \say{They fought the regime with other brigades, they lived among the people, they went up to the pulpits and gave the Friday sermon, they gave lessons and lectures, they have media activities, they fought in the frontline and gave martyrs and leaders}~\autocite{dareih2015unmasked}.

\subsection{Internal Demographic Pressures}

The group began to attract an increasing number of foreign fighters, and particularly those motivated by the Nusra Front’s status as a Syrian branch of the Islamic State of Iraq. By the end of 2012, it was common for reporters and analysts to describe the Nusra Front as a largely-foreign force. For example, a November 2012 article from France24 described the Nusra Front as “made up of mostly foreign militants from Iraq, Saudi Arabia and Central Asia” who were willing to coordinate with local Syrian militias but were ultimately working for an Islamist project\autocite{fvc2012jabhat}. Among the motivations described by foreign fighters arriving in the first waves, from 2012-2013, were desire to transfer frustration over local repression into a battleground where they could be active. This motivation is particularly closely associated with Chechen fighters, who quickly became significant military actors. Speaking to Joanna Paraszczuk, the commander of a private jihadi military contractor in Syria reflected on his arrival in Syria in 2013 and summarized this perspective. Noting a longstanding frustration with the lack of opportunities for military action in Chechnya, he said:

\say{...in 2012-2013, news from Syria was widely disseminated. The situation in Syria was extremely clear, and reminiscent of the situation in Chechnya during the war. This really caught my interest, all the more so because at the time in Syria there were Chechens who were taking part in the fight against Assad. I had two choices: to endure and wait for a suitable moment to take action in my homeland, or go to Syria to help the Muslims against the bloody regime of Bashar al-Assad. I consulted with the brothers, and I decided to go to Syria.}(http://www.chechensinsyria.com/?p=26323)

 
During this early waves, there were indications that the Nusra Front was intentionally focusing on recruiting foreign fighters from Arab communities, to ensure that the fighters would have sufficient language skills to be able to integrate with their command structure. While this decision enhanced military cohesion and battlefield efficacy, the unintended consequence may have been to shift the demographics of the foreign fighters towards a population inclined towards being abrasive with the Syrian community from whom the Nusra Front would need to ally with and derive support from. In one recollection compiled by Vera Mironova, a Syrian doctor complained about Arab foreign fighters being particularly harsh towards the local community.  He began by praising non-Arab foreigners for “being nice to the people” and investing in good-faith gestures, such as trying to communicate despite imperfect Arabic. In contrast, the doctor associated Arab foreign fighters with a disrespectful and harsh relationship to the local community, adding: “Tunisians, Yemenis, and Saudis, on the other hand, were so mean and harsh on people. They used to enforce the rules by [threatening with] weapons, and if you didn’t listen, they would arrest you. They didn’t discuss anything, while the non-Arabs often initiated discussions.”\footnote{Although the doctor’s recollections were primarily about ISIS foreign fighters, and the Nusra Front subsequently went to great lengths to limit potential friction points between their remaining foreign fighters and the local community, this recollection  could have equally applied to the period before the split.}~\autocite[159]{mironova} 


In the same year, Nusra Front local coordinators described facilitating foreign volunteers and weapons from Lebanon into Syria through the northern countryside of Holms, with one bragging that “I have sent in brothers from Saudi Arabia, Iraq, Pakistan, Lebanon, Turkmenistan, France and even from Britain" who join the Holms branch of the Nusra Front or link up with other groups within Jabhat al-Nusra to conduct combined operations~\autocite{sherlock2012inside}. Along with the discussion of foreign troops, who were more willing to introduce aggressive tactics, such as suicide bombings, that allowed the Nusra Front to raise their profile and maximise their effectiveness, the leader of the Nusra Front brigade in Holms described the Syrian conflict using the same religiously-tinged language and framing of the group and the conflict, calling the group “the sword of the Islamic land"~\autocite{sherlock2012inside}. The description if notable for, firstly, using a metaphor that speaks to a historical and transnational Islamist literature tradition and, secondly, for describing territoriality based on religion rather than a Syrian political identity. 

The Holmsi Nusra Front leader issued a general invitation for Muslims to join the group’s ranks:

\say{Any pure Muslim can join Jabhat al-Nusra, but they have to be committed to Allah and fighting only for Allah." To demonstrate religious motivation, the leader cited a common shibboleth used by jihadi fighters: willingness to cease smoking.\footnote{During the conflict, refraining from smoking became a common point through which jihadis regarded themselves as morally and physically superior to non-jihadi fighters.} The leader in Holms insisted that recruits should show their commitment by refraining from smoking, adding that: “If they are smokers and they die in Syria, how do we know that they died for God – and not because they were trying to go to reach a place to buy another packet?}~\autocite{sherlock2012inside}


Throughout rebel territories as the conflict continued,  foreign fighters were instrumental in effectuating the aggressive tactics that boosted the reputation of the Nusra Front as one of the most effective fighting units in the conflict. In particular, as in Holms, the Nusra Front was able to leverage vehicle-borne improvised explosive devices (VBIEDs) driven by suicide bombers to soften military bases for subsequent waves of ground troops. These suicide bombings and the bravery of their fighters in battle made the Nusra Front a desirable group for alliances and cooperation among the anti-regime forces, which then bolstered the Nusra Front’s local profile and influence~\autocite{lister2016profiling}.  The first of these suicide bombers were reportedly foreign fighters, may from the conflicts in Iraq and Afghanistan although quickly matched by recruits from within Syria~\autocite{ignatius2012affiliate, sherlock2012inside}. 

Foreign connections were also used by the Nusra Front to generate flows of recruitment and financing through existing global jihadi networks~\autocite{sherlock2012inside}. As well, the money that foreign fighters brought with them was 


These new fighters jeopardized the Syrian identity of the Nusra Front, and threatened to make the Front’s membership Syrian-minority \autocite{adraoui2019case}.  For example, one of the first interviews between  a Western journalist and a member of the Nusra Front was published online in March 2013. The interview was prefaced with a discussion of the difficulty of obtaining interviews from Nusra Front members but that access had been easing “as more Syrians join the group and they make more gains on the ground in the fight against the Syrian government”~\autocite{marrouch2013nusra}. Nevertheless, the international membership of the early Nusra Front was highlighted by the identity of their interviewee who claimed to be a 21-year-old from Libya and who described his “vision for Syria” using a framing and vocabulary that made no reference specifically to Syria:

\say{We Muslims have a certainty, there is talk of the prophet who preached that the best place on earth is the Levant. He also said that God chooses whom he wishes to be his best followers to be from here. I hope God chooses me and this is why I came to here, too.}~\autocite{marrouch2013nusra}

As the Nusra Front gained in prominence through both their military and propaganda efforts, they began to deepen the recruitment from native Syrians, which introduced tension over preferred goals. According to interviews with fighters at the time, foreign members of the group pushed to violate the leadership’s polices against striking non-Muslim targets. One fighter reported internal tensions by noting that while native Syrian members were willing to follow al-Qaeda’s stated policy of avoiding sectarian violence, \say{some of the foreign fighters hate the west and all non-Muslims… They want to attack churches. Personally, I don't like this. But this is how they were taught in Iraq and Chechnya.}~\autocite{sherlock2012inside}.

As their profile rose in the conflict, the transnational origins of the Nura Front were revealed when al-Qaeda in Iraq  (AQI) announced that the leadership of the Nusra Front had been sent into Syria to establish a branch of the transnational jihadi insurgency in the country. In the announcement, the leader of al-Qaeda in Iraq, Abu Bakr al-Baghdadi, ordered the Nusra Front and Jolani to reintegrate with the Iraq-based jihadi group. Jolani refused the order, announced the Nusra Front’s independence as a Syrian militant group, and announced that the group would answer only to al-Qaeda’s central leadership. This lead to several years of bitter conflict between al-Qaeda in Iraq’s newly regionalized Islamic State of Iraq and al-Sham (ISIS or ISIL) over which group was the legitimate physical and spiritual leader of the jihadi resistence to Bashar al-Assad.  In particular, ISIS espoused an explicitly transnational mission of social and political revolution, with the Nusra Front retaining a stronger focus on regime change in Syria. 

\subsection{Nusra Front Syrianization Campaign}

The recruitment and personnel trajectories of the Islamic State and the Nusra Front began to diverge dramatically, with IS continuing to issue transnational calls and the Nusra Front increasingly trying to assert a Syrian identity. 

At the outset of the Nusra-ISIS conflict, fighters and brigades would switch affiliation, often articulating their preferences for the goals and tactics of one group over the other.  In dissident messages, Nusra Front appears to have had considerable internal tensions over whether, and how, to be a Syrian anti-Assad movement or a jihadi group based Syria. Foreign fighters defected from the Nusra Front to the Islamic State because they were more attracted to to the transnational goals of the Islamic State. One common sentiment among fighters who described their reasons for switching from the Nusra Front to the Islamic State was that \say{Jabhat al-Nusra was only interested in fighting Assad while ISIS was dedicated to building an Islamic state and enforcing sharia law}~\autocite[127]{mironova2019freedom}.

 These tensions are attested to in reports from dissident fighters who left the Nusra Front fror ISIS, in part due to the defector’s preferences to fight for an Islamist social revolution. The dissident British fighter, Omar Hassan who wrote under the name “Abu Sa’eed al-Britani” issued a six part statement of \say{why [he]  left jn} to social media outlets in 2015. Central to his sense of disillusionment with the group were his complaints about Nusra Front prioritizing their own military goals instead of aiding the other Islamist fighters [pt 1], and the  NF’s apparent lack of interest in pursuing islamic governance, and their \say{fear of being labeled extremists} [pt 2].  In the second of his six-part messages, he describes frustration with the NF’s unwillingness to simultaneously launch military and social campaigns: \say{they said that Dawah was needed to be given to the people before we could judge them according to the Shariah, yet they were not so keen to give Dawah...So the fruits of Jabhat's jihad and struggle was nothing more than removing the tyrant}[pt 2]

The dissident message continued, inadvertently revealing some of the difficulties of managing an influx of recruits motivated by goals other than the main goal of the movement. Notably, the third installment of the dissident message describes at great length the British fighter’s frustrations that the Nusra front was unwilling to impose an extremist social policy on the Syrian population. At the same time, the fighter believed that participating in the conflict empowered him to impose his social preferences, seething in frustration that despite being armed he was being restricted from imposing hardline islamist behaviors. The third installment of his message heavily emphasized his belief that being armed should have permitted him to impose social preferences and castigated the nusra front’s social policy and their restrictions on fighters. In the short essay, he emphasizes repeatedly the sense of power that being armed gave him and his frustration that he could not use the arms to implement social policy;

\say{When I first arrived in Sham and walked the streets of Atmah, Dana, and its surrounding areas, I felt an immediate kick of excitement that I was in the land of jihad and carrying a gun gave me a sense of honor. However, as the weeks went by, the initial excitement faded away, I started realising that there was a lot of vice and munkar around me..I started feeling as if I was back in Edgeware Road in London. The only difference being, that I had a gun… You couldn't even tell someone to stop smoking even though we had a gun. Hardly anyone in the town had respect for Jabhat An-Nusrah and their approach to implementing the Shariah was weak and futile.} [pt3]

This theme that the Nusra Front shedding fighters with preferences that were more extreme than the NF’s preferences also feature in other research into the personnel management policies of jihadi groups in the conflict. [[Use some of the Mironova quotes]]

\subsection{Nusra Response}

As the Nusra Front began seeking independence and distancing itself from both the al-Qaeda and Islamic State transnational movements, the group’s leaders appear to have continued to encourage a Syrian-first perspective in the movement. By 2016, the Nusra Front was continuing to position themselves as one among the nationalist rebel movements. They began heavily recruiting from among the opposition communities in Aleppo and Idlib, leveraging perceptions that the international community had abandoned the Syrian people and cause. This recruiting paid off dramatically, with reports that the Nusra Front gained at least 3,000 Syrian forces in the first half of 2016~\autocite[6]{lister2016profiling}. In 2016, the nusra front tried to unify the Syrian opposition around them, with the nusra front as the leader. This effort resoundingly failed, with prospective subsidiaries balking at the nusra front’s connections to al-qaeda %(https://ctc.usma.edu/app/uploads/2018/02/CTC-Sentinel_Vol11Iss2-2.pdf)

 On July 28, 2016 the Nusra Front rebranded themselves as the Jabhat Fateh al-Sham (“Syria Conquest Front” or JFS ). Amid the re-branding announcement, JFS positioned themselves as a local movement, with “no affiliation to any external entity” and described their intention to coordinate more closely with other opposition rebel groups in Syria~\autocite{jabhatfs2016announce, jabhatfs2016break}. The rebranding was widely believed to be a strategy to increase their grassroots support, and was accompanied by relaxing their typically “long and strict recruitment procedures” in order to add more flexibility and attract more local fighters~\autocite{haid2017behind}.

Interestingly, and suggesting that Jabhat Fateh al-Sham was still conscious of the influence of their foreign audience, on the same day as the announcement of their rebranding as Jabhat Fateh al-Sham, and among persistent rumors that the group’s leadership was meeting to decide whether or not to break away from al-Qaeda’s transnational leadership, the Nusra Front/Jabah Fateh al-Sham released a second video in which a speaker identified as the “deputy of [al-Qaeda leader] Ayman al-Zawahiri” was featured telling the Syrian branch to take any course of action needed to safeguard the Syrian jihad. In the video the alleged deputy counselled:

\say{... we direct the leadership of the Nusra Front to go ahead with what preserves the good of Islam and the Muslims, and protects the jihad of the Syrian people. We urge them to take the appropriate steps towards this matter. This is a step from us, and a call from us to all the factions in Sham to unify in what Allah approves of, and to work together… Be as one rank that protects our people, defends our lands, and please our eyes with your unity by meeting on a wise Islamic government that restores truth and spreads justice among the Muslims. Allah permitting, we will be the first to support it.}~\autocite{jabhatfs2016deputy}

By late 2015, the conflict continued to internationalize with Russian intervention to support the Assad regime. Relationships between rebel groups became increasingly critical for survival adding further importance to the Nusra Front’s strategy of trying to coordinate and lead coalitions of Sunni rebel brigades, “fostering a constructive relationship of dependence…[and] pro-al-Qaeda support groups” among the armed opposition~\autocite[33]{lister2016}. Moreover, no Sunni rebel group was strong enough to control the liberated territory and consolidate the myriad funding and logistics routes, least of all the Nusra Front who lacked a single state sponsor and who therefore tended to source their weapons and ammunition from booty captured during battle and from taxes imposed on allowing other groups’ suppyl routes to operate [Lister 2016, pg 32].

%% [[need a quick summary of 2016-2019, to bridge the part where I skipped ahead three years...By 2017, the Nursa Front’s grassroots were increasingly Syrian,...]

As with other groups in the conflict, HTS shares a continued urgency for manpower, and have issued repeated calls for Syrians to join them, including members of the Syrian diaspora. That the recruitment calls specifically focus on members of the Syrian diaspora to return is a notable difference from the aggressive international recruitment platform of the early years of the conflict. For example, in October 2019, the group released a video message, titled “Strive With Your Life: Video Nasheed,” that called on members of the Syrian diaspora to return to Syria and join the fight.  As the following excerpt highlights, the nationalistic framing of the video is in striking contrast to the framing of previous messages from Nusra Front recruiters.  Instead of presenting the conflict in transnational or religious terms and framing their appeals to a broad audience, the October 2019 message tries to make a connection with the specific and personal connections of native Syrians with their homeland. While recruitment messages targeting foreign fighters often highlighlighted the purported openness and accessibility of Syria and the theoretically-transnational Muslim community, the 2019 message goes in the opposite direction by staing that Syria is a battleground to which members of the diasora are already a part by virtue of their nationality:
"This is a message that I am directing to our beloved young people and our heroes. Our heroes who have left the fronts, and they stayed in the camps in cities and towns. Until when you avoid the land of glory, honor, dignity, and faith. These lands of heroism and knights, and you are its people. Until when you are away from these lands. These lands that whoever is away from them will have regret. Until when our heroes, why are you away from these sites?
“Until when young people will you fill the camps with those holding back and sitting and women. These are the places of women, and here are the lands of men and of glory and honor. These are the lands that Allah the Almighty loves and is satisfied with its people, and He honors them with rewards and giving.”~\autocite{hts2019join} 
In the same year, Abdullah al-Muhaysini, a Saudi cleric based in Syria called on Syrians in Europe to return to the country to fight. Muhaysini framed the motivation of foreign recruits as having journeyed to Syria to “be behind” Syrian revolutionaries, “defend our land which is your land” rather than to “rule or to take any position here”~\autocite{muhaysini2019capital}. Directing listeners to reach out on social media for instructions on how to come back, Muhaysini framed incentive to return as one of nationalism and homeland:

\say{you will have remorse, and you will say I was in my country why did I leave it? Is this the reward for my country, is this the reward for my family, is this the reward of my imprisoned brothers?}~\autocite{muhaysini2019capital}


%% Smooth this out:

We can also see variation in success in restricting movement:  with the well-publicized ISIS lockdowns and hunting of defectors on one extreme and then rapid shifting among FSA and pre-ISIS jihadi accounts on the other..

Variation in socialization capacity. Isis on one extreme, again, due to territory they control, centralized policies for outside access to information (eg: capture cellphones). 
I keep saying that I have references to camps getting shorter and shorter within NF: so somewhere in the middle, they have training camps but are sensitive to changes in the operating space. 


\section{ISIS: Decentralization and internal heterogeneity}

Although ISIS was very aggressive about limiting exit opportunities, which should give the leaders more ability to control the group, they also had a very strong brigade culture, which seemed to serve as a hub for the political, social, and ideological variation among fighters [eg: Mironova 2019, 130-133 on homophilly in the brigades. Homophilly in the foreign fighter brigades is a way in which the group never imposed uniform ideological preferences internally, with the result that different brigades were virtually even fighting different conflicts. Thus, for example, Chechen brigades primarily motivated by fighting russia could insist on being deployed on fronts where they would fight Russians in syria [mironova ].

\say{Professional fighters who had gone to Syria left when another conflict they were interested in started elsewhere. For example, one Chechen mercenary who had trained opposition groups in Syria at the beginning of the conflict left for the Ukraine when Russia invaded in 2014 and started his own battalion to train Ukrainian forces...Fighters of other nationalities, particularly ones from Central Asia went to Afghanistan when it became clear that ISIS in Syria and Iraq would fall.} [Mironova 2018, pg 135]



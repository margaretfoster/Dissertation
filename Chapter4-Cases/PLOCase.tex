
\epigraph{We in the PLF [Palestinian Liberation Front]... felt as if were caught, as the old Arabic saying goes, between a hammer and an anvil: the hammer was the conventional [Chairman's] policy and the anvil the newly born struggle of the Palestinian fedayeen--- \textup{Shafiq Al-Hout}, My Life in the PLO}

In the mid-1960s, the founders of the Palestinian Liberation Organization (PLO) faced a dilemma. They wanted to create an umbrella organization for the Palestinian national movement. However, militancy had become more popular than the leftist Nasserite ideology preferred by the PLO's Executive Committee. Even worse, members of the Executive Committee believed that the \textit{fedayeen} militant groups were rushing into a confrontation with Israel that the Palestinian nationalist movement was not ready for and which would likely result in a serious setback for the Palestinian cause.  As PLO co-founder, Shafiq Al-Hout recounted in his memoir, the Executive Committee tried to unify the Palestinian movements without adopting the aggressive agenda of the militant groups. Initially, the Chairman, Ahmed al-Shuqayri, \say{tried to absorb the inter-Palestinian problems by forming a new Executive Committee, which was intended to bring the two generations---the traditional bureaucratic one and the young revolutionary one which was keen on initiating new practices---and to be capable of absorbing the factions that were still refusing to participate}~\autocite[53]{alḥout2011my}. Within two years, however, the Executive Committee was obliged to accommodate the militant factions after \say{debate intensified between the conventionalists and the newcomers} and \say{al-Shuqayri was forced to accept new members of the Executive Committee from the younger generation}~\autocite[54]{alḥout2011my}.

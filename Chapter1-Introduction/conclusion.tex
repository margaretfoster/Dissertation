\chapter{Conclusion}

In the previous chapters, I proposed an under-examined form of internal change within organizations. The process, a \say{personnel resource curse} occurs when recruits are brought into an organization to strengthen the organization's short-term outlook, but ultimately exert internal pressure that transforms the movement. The dissertation proposed a mechanism for this transformation by proposing a negotiation-based model of leader decision-making and delegation. The manuscript opened and concluded with a general framing of the theorized recruitment, accommodation, and transformation process. It then turned to the context of militant group operations and the dynamics of conflict in order to unpack the implications and outcomes of the recruitment, negotiation, and accommodation process. The militant context is particularly important, as the evolution of preferences of militant groups have outsized influence on the lives and security throughout their zones of operation. Using the ongoing civil war in Syria, the dissertation analyzed how militant leaders attempted to manage the complexities of preference heterogeneity among their rank-and-file. The dissertation then used the public messaging from al-Qaeda's affiliate in Yemen to examine changes in self-presentation that follow from shifting recruitment patterns.

For many organizations, recruitment has profound downstream influence.  In some cases, this can take the form of a "personnel resource curse" when an influx of new recruits brings strength and capacity but also generates upward-driving pressure to change the group's operations and objectives. In asking how, and under what conditions, recruitment shocks exert transformational pressures on organizations, my research agenda addresses a facet of organizational evolution and resilience that is prevalent yet underexamined theoretically. Although understudied, this is an important research area, as understanding the impact of grassroots-driven and bottom-up transformational pressures on the evolution of organizations has broad implications, from philosophical questions about how organizations maintain their identity and priorities, to  to tactical conclusions about how to best nurture or combat organizations undergoing internal transformations.

This dissertation has developed a general theory of how organizations evolve in response to changes in the makeup of their membership. 
The second chapter of my dissertation developed a general theory of membership-driven transformation. It presented a mechanism of leader decisionmaking in organizations that operate under adverse conditions, such as resource scarcity or state security pressure. These organizations walk a narrow path of survival and therefore more often find themselves being forced to trade operational best practices for short-term survivability.

The third chapter focused more closely on the implications of the theory for militant organizations  that can be applied in contexts ranging from militant groups struggling to operate and expand despite extensive state repression, to issue-based organizations that seek an influx of resources and skills, to decentralized movements that lack strong mechanisms of grassroots control.  Although operating in separate domains and with  different goals, strategies, and tactics, the common thread of resource scarcity and distance between grassroots and the leadership means that each of these types of organizations is susceptible to a sudden change in their recruitment having follow-on effects through the organization. The fourth chapter continued the focus on militant groups, by presenting an extended study of the effects of jihadi foreign fighters in the Syrian Jabhat al-Nusra. The focus on a single militant groups brought a more concrete illustration of not only the internal tensions caused by heterogeneous preferences between the base and leadership, but also the holistic view of how the actions of leaders, recruits, and the ecosystem in which organizations operate can shape the trajectory of pressure for change. 

The fifth chapter of my dissertation begins from the question of how outside observers might discover a bottom-up transformation occurring
in an opaque organization.  The chapter used the example of al-Qaeda in the Arabian Peninsula (AQAP) to explore the externally-visible implications of a change in the recruitment demographics of the militant group. AQAP is a particularly interesting case
of the theory, as the jihadi organization was widely reported to have experienced a recruitment boom after 2009, when local and
international politics meant that they were suddenly able to engage in widespread recruiting from local Yemeni Sunni tribes.  This new base
was desirable for AQAP as it brought local legitimacy, protection, and resources, but potentially introduced a mismatch in priorities between
AQAP's leadership, dedicated to transnational jihad, and their new, more parochial, base.  Moreover, AQAP created their own comparison
case, in the form of a spin-off organization, Ansar al-Sharia, which was reportedly intended to represent local facets of the group's
operations.  

 The final chapter featured a survey of non-profit leaders and managers and finds that nearly half of the sample experienced a need to recruit staff for skills rather than alignment with the organization's mission and 20\% of the sample acknowledged that accommodating these new members resulted in changing the group's mission.

The overall project suggests several lines of research moving forward, particularly focusing on the direction of changes and the strategic interaction between leaders and their followers. I briefly outline three future avenues.

First, one future avenue is to develop a dataset of the bottom-up transformation theory as applied to militant groups. Unlike the vast universe of licit organizations that might experience a recruitment shock, the universe of rebel group---particularly those that have gained enough strength to control territory. This data can be matched with open-source data, derived from news archives, historical texts, U.N. reports, and declassified documents, to record which groups have experienced recruitment shocks. This data can thus be used to evaluate the predictions of when militant leaders would be expected to recruit large cohorts and when they should be expected to accommodate. 

Second, I expect to continue move towards measuring micro-level recruitment changes and real-time leader decision-making. Thus far, the dissertation has proposed a general theory, demonstrated through an original survey that the underlying claim about rapid growth patterns and downstream consequences are experienced in at least one context, pulled evidence that this is a challenge faced by leaders.  We are currently in an information environment that should allow for this type of investigation: the data revolution is encouraging licit organizations to maintain more and richer data about their operations. This operations data can provide additional insight and quantitative measurement of the downstream effects of demographic changes. On the clandestine side, the civil war in Syria may be a harbinger of a continued explosion of real-time information from conflicts around the world.\footnote{Although Syria may be an outlier in this respect, as it has been in so many other conflict attributes, increasingly available data monitoring and telecommunications can continue to allow for insight into the decision-making of commanders as well as internal frustrations that can help to indicate the direction of pressures. Complaints from dissidents are a useful indicator of the direction in which a leader does not want to go. Thus subsequent movement in that direction is unlikely to have been the result of leader preferences.}  

% Additional cases and downstream consequences ()
A third future avenue is to increase the focus on the strategic interplay in which leaders attempt to respond to, and minimize, the internal leverage of their grassroots.  I particularly intend to develop additional case studies and seek leader insights for how they weigh strategies to guard their own leverage with the constraints of capacity, resources, and--- for clandestine leaders--- security. I intend to continue to develop cases to analyze how leaders try to restructure in the face of internal pressures are also worth investigating, as well as the degree to which these strategies are successful. 

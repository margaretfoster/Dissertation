\chapter{Introduction}

%%Type your introduction here.  This is ``technically'' your first chapter of the dissertation/thesis.

Recruitment opportunities that seem to improve the strength and resilience of an organization may profoundly reshape the group.  This manuscript argues that opportunities for rapid expansion can strengthen an organization in the short run, but undermine the leader's ability to pursue their goals by introducing internal pressure to satisfy the priorities of new recruits. This pressure may initiate a process of internal transformation such the organization is reshaped according to the preference of the recruits rather than the original leaders. In theorizing a recruitment-driven process of organizational change, I analyze an underexplored, yet widespread, trade-off that organizations face as they grow.  This process of bottom-up transformation differs profoundly from the accepted wisdom among researchers and practitioners. The conventional narrative of organizational change and the origins of organizational culture holds that both originate in the top echelons of an organization~\autocite{hambrick1984upper, schein1990organizational}.

The central theoretical claim of this dissertation is that when resource-constrained organizations grow quickly, expansion strains the organization's selection, socializing, and monitoring capacities.  With neither the time to change the recruit's preferences via socialization nor the capacity to reinforce institutions that monitor and enforce compliance, leaders manage their rank-and-file via compromise. This grants recruits retain leverage over group decisions about strategies and tactics.  Thus, the relationship of the group's leaders to the rank-and-file increasingly takes the form of a negotiation and results in leaders incorporating the recruits' original preferences into the group's mission. The process of transformation rests on a series of interrelated pathways:

First, strategic leaders can be induced to recruit new members whose preferences do not align with those of the leadership and in cohorts that exceed the organization's socialization capacity. I demonstrate that this claim is well-founded in Chapters 4 and 6, by using qualitative case studies as well as an original survey of leaders. 

Second, horizontal delegation requires consent and buy-in.  This consent-based view of delegation implies that the relationship between a leader and their subordinates is based on negotiation. The project synthesizes the inflow of recruits and the negotiation view of delegation into a model of leader accommodation to pressure from incompletely socialized recruits. Chapter~\ref{chapter:theory} explores this dynamic in a general case, while Chapter~\ref{chapter:militants} analyzes the implication of this framework in the context of militant groups who operate in the face of repressive pressure. Finally, Chapter 5 highlights the leverage that the theory provides to augment understanding of even difficult and opaque organizations. 
 
Although previous scholars have emphasized the impact of internal
politics on organizational functioning, both generally and in specific
domains, the mechanisms of accommodation and adjustment are often
either elided or presented as idiosyncratic to specific domains.
Elsewhere, the dominant approach in the substantial literature on
organizational change is to evaluate changes initiated by an
organization's leaders. Yet, both case studies and personal anecdotes
about small or precarious organizations  often reveal that leaders of
organizations are often more constrained than the literature
assumes.

When there are tight control mechanisms and powerful internal institutions,
leaders can ensure that their new recruits implement the actions that the
group’s leadership wants. Training and indoctrination procedures can
reorient the preferences of the membership base towards those of the
leadership. However, in the absence of strength in either, or both,
areas, leaders can easily lose control over their movements. The
process is especially dangerous if leaders find themselves recruiting
from already-cohesive populations or recruiting under heavy
repression.  This observation suggests that what appears to
be a process that brings in much-needed strength and resources can
create internal strains and transformative pressures.

%% Odds and ends to maybe keep:

In asking how, and under what conditions, changes in human capital exert pressure on group leaders to transform their organization, this dissertation presents a model of upward-driving pressure for organizational change that identifies and fills a gap that spans several research areas. Existing work on organizational change tends to focus on leader-driven and top-down changes (see, for example,~\cite{armenakis1999organizational, hannan1984structural, kanter2003challenge, fernandez2017managing}). Conversely, the literature on social movements emphasizes the agency of recruits, but tends to downplay the role of leaders and organization leadership~\autocite{campbell_2005, morris2004leadership}. Research on mission drift addresses the dynamics that drive organizations to change in ways not intended by the leadership, but both academics and practitioners tend to emphasize material influences rather than personnel changes (\textit{i.e.}~\cite{cornforth2014understanding, jones2007multiple, schleckser2015inc}). Yet, although bottom-up transformations have been largely overlooked by scholars, the histories of many militant and non-militant groups feature organizational changes initiated and driven by recruits.

This dissertation makes important empirical contributions by theorizing and
testing an under-explored vector of change in organizations. By introducing a general mechanism, this project contributes to scholarship on organizational change. In exploring the specific implications for militant groups, it also contributes to the literature on the organizational dynamics of conflict actors.  


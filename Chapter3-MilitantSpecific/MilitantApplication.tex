\chapter{The Personnel Resource Curse For Militant Groups}
\label{chapter:militants}

%why does this theory bring particular leverage to our understanding of militant groups

In this chapter, I apply the analytical leverage of the accommodation model and personnel resource curse identified in Chapter~\ref{chapter:theory} to the operation and evolution of particular type of politically-salient yet often-opaque organization: militant groups. I illustrate the transformation dynamic via the trajectory of several militant groups that experienced internal pressure for strategic and tactical changes after a recruitment shock. As I argue below, the personnel resource curse and the accommodation mechanism introduced have particular implications and application to the operation of clandestine and revolutionary militant organizations.  

Research into the disciplinary and socializing institutions enacted by clandestine and militant organizations emphasize that the identity, actions, and perceptions of individual fighters can have significant consequences for the organization~\autocite{beber2013logic, cohen2013explaining, green2015commander, kalyvas2006logic, schubiger2017ideology, staniland2014networks,  weinstein2006inside}. These combine to suggest that militant groups face a stark trade-off in that integration with local communities provides resources and protection but can undermine broader goals by leaving the organization beholden to the parochial interests of local recruits.

The personnel resource curse is particularly salient for militant groups because the constraints that they operate under have consequential second-order effects on their internal functioning.\footnote{I use the term \textit{militant group} to capture the sense of an armed organization that is subject to repressive pressure from state actor(s) and/or other such armed organizations. I primarily focus on ideologically-motivated, and particularly revolutionary, groups because it is relatively more straightforward to track changes in goals and strategies over time. Although not addressed as such, the theory should apply to non-ideological armed groups such as smuggling networks or violent gangs, conditional on these groups also exceeding their socializing bandwidth.} As outlined below, these constraints limit the leader's ability to establish institutions that mitigate pressure to accommodate the preferences of recruits but also increase the pathways through which recruit behaviors shape the pool of future recruits.

Recruitment shocks that are not matched with strong socializing capacity can generate competing internal constituencies who then attempt to monopolize the future direction of the group.\footnote{See \cite{mosinger2017dissident} for a treatment of this dynamic in the Nicaraguan FSLN Movement and \cite{perkoski2015organizational} for discussion of similar dynamics among Irish Republican militias.} In the absence of strong socializing or monitoring institutions, the pre-existing motivations that the new recruits bring into the organization are likely to inform the actions that are enacted under the auspices of group activities.

Competing internal constituencies and factions often  result in violence and military operations intended to advance the interests of that internal constituency rather than the organization as a whole. Such interests may relate to the current conflict dynamics or to expectations about relative positions after the conflict has concluded~\autocite{balcells2010rivalry}. Either way, such actions seed knock-on effects in subsequent periods of conflict as they are the actions that the community at large will ascribe to the organization. Thus, the actions of recruits inform the expectations about the organization's goals that other actors will use to shape their own involvement in the conflict. 

\section{Mapping the Theory to Militant Groups}

Militant organizations are particularly susceptible to experiencing a success trap caused by the bottom-up transformation process. To highlight the connection between the general theory presented in Chapter~\ref{chapter:theory} and the particular circumstances faced by militant groups, this section translates the parameters of the formal model into the context of militant operations. The importance of the socialization step provides a natural point at which to group the parameters for analysis:  those that influence decisions that occur before socialization, the parameters that influence socialization, and parameters that influence gameplay after socialization.  % Overall, the changes should decrease the leader's position.  Add also that cost of information can go way up.


The parameters that influence the first two decisions of the game are the leader's penalty for group failure, the recruit's benefit for being part of a group, and the cost to the leader of obtaining information about the potential recruit(s).  In each case, operating in a conflict setting can be expected to increase the value of these parameters. Armed revolution is an inherently risky prospect with often-deadly consequences for group failure~\autocite{lichbach1998rebel}. This can be expected to increase the pressure on both leaders and recruits. For the leader, high cost to failure broadens the range of situations in which a leader can be induced to admit a large flow of recruits to bolster their short-term position. For potential recruits, conflict can increase $U_{G}$ by making it appear to be safer to be part of an organization~\autocite{kalyvas2007free}. Finally, conflict increases the difficulty-- and thus cost---of gathering information about the motivations of potential recruits~\autocite{weinstein2005resources}. Indeed, scholars have documented significant investment in mechanisms that reduce this information costs by relying on participation in subcultures and social networks that are costly for recruits to maintain~(\textit{e.g.}~\cite{hegghammer2013recruiter, forney2015can}).

In a conflict setting, leaders may lack the ability to implement organizational structures and institutions that augment their socialization abilities. Initial structural patterns --- often driven by recruitment needs--- often have long downstream legacies, as militant groups are very difficult to restructure after conflict starts~\autocite{staniland2014networks, sinno2010organizations, weinstein2006inside}. Thus, militant leaders have fewer options to restructure their organizations to re-balance their leverage vis-a-vis their (theoretical) subordinates. Instead, logistical, operational, and security considerations often take priority.\footnote{For a discussion of how operational context can drive militant group structure, see the analysis of organizational variety among Afghan militant organizations in the 1980s in \cite[130-134]{sinno2010organizations}}
 

Finally, the conflict context can be expected to act on the tie between probability that the action will succeed and the resources that the recruit brings to the group.  During periods of high-risk mobilization, community ties and social networks often provide critical resources and support for the movement~\autocite{arjona2016rebelocracy, malthaner2015violence, parkinson2013organizing, staniland2014networks, wood2008social}. If, as argued above, militancy could be expected to reduces the leverage afforded to group leaders, the same context affords more internal leverage to recruits.

%% I think that this needs to move, but I can't figure out where it goes.
 %%Three circumstances make a leader more likely to be pressured to accommodate their grassroots: (1) labor mobility among the rank-and-file; (2) differences in priorities of the leadership and new base, and (3) restricted socialization capacity. In particular, the socialization limitation can follow from restricted ability or desire to titrate the inflow of new members.   
 
\section{Effect of Labor Mobility}

Labor mobility is necessary for the organization's rank and file to be credibly able to exit if the organization does not accommodate their priorities, while difference in priorities and preferences are required for there to be an internal tension between leaders and the
grassroots.\footnote{Another way to conceptualize the absence of interest divergence between the leaders and the members of their organizations is to consider an organization without a strong attachment to a particular set of goals. In this case, we should expect a grassroots-driven transformation to occur quickly, as the leadership does not undergo internal pushback against modifying their organizations to capitalize on the preferences of a desirable pool of recruits.} The third assumption, restricted ability or desire to limit inflow, is an avenue through which leaders are prevented from effectively socializing the new entrants. Either the leader fails to foresee the potential for tension or, more likely, circumstance or optimism cause the leader to discount the effects of the internal tension. The latter case reflects the counter-intuitive expectations of the formal model, in which the rational leader is able to foresee that recruiting will result in accommodation but prefers accommodation to failure.

As discussed in the previous chapter, survival threats provide one powerful reason for a leader to discount the effects of tension between their goals and the goals of their new membership.  A leader who wants their group to continue to exist in the short-run may be tempted to discount the long-term effects of admitting new members whose preferences are similar to, but not exactly in alignment with, their own. Thus, leaders operating in precarious contexts, such as leaders of fledgling organizations or organizations that operate in high-failure domains, should be much more likely to both become subjected to bottom-up transformation as well as to perceive themselves to be playing the accommodation game.
    
Bottom-up transformation is driven by the threat that dissatisfied members can leave the group. In order to have the internal leverage to force the inclusion of their goals, the rank and file must to be credibly able to exit if the organization does not accommodate their priorities. Similarly, the leader must be invested in keeping the recruits. Thus, labor mobility is a necessary condition for leaders to be induced to move their groups away from their own goals and towards the preferences of their grassroots.  Transformation occurs when the leader's interest in keeping the recruits is such that they are willing to compromise other goals or best practice in order to keep the recruits from leaving. 

Mustfa Hamid, an advisor and jihadi strategist involved in the early years of al-Qaeda, articulated how worry about losing recruits can induce leaders to grant self-defeating concessions to rank-and-file members. Reflecting on the quality of Arab foreign fighters in Usama bin Laden's Afghan camps in the late 1980s, Hamid observed that trainees would lobby bin Laden to reduce the intensity of the camp. Hamid noted: \say{...discipline was a serious issue but [trainers] Abu Ubaydah and Abu Hafs did a very good job under difficult circumstances. They tried to control the youth and train them in the severe way that creates discipline. Then the youth would go to Abu Abdullah [bin Laden] and he would be gentle with them, perhaps because he feared they might leave--- and the work of Abu Ubaydah and Abu Hafs became harder}~\autocite[99]{hamid2015arabs}. As a result, the trainees were \say{woefully unprepared} for a critical May 1987 battle and were ultimately saved by luck and the intervention of local allies~\autocite[100]{hamid2015arabs}.

Leaders have reason to be be concerned about the possibility of critical recruits and units leaving their group as group exit, including defection and fragmentation, is a common feature of organizational histories~\autocite{bakke2012plague, mclauchlin2010loyalty, seymour2014factions, pearlman2012nonstate}. The possibility that rank-and-file fighters can defect from their groups to others operating in a similar ideological and physical area can be consequential for the evolution of a conflict. For example, frequent movement of personnel between brigades and units as fighters try to improve their access to material, resources, leadership, and preferred ideological orientation has been identified as a key feature promoting the transformation of Sunni resistance to the regime of Bashar al-Assad in Syria from a loose collection of ideologically moderate militias to one dominated by Islamist factions~\autocite{mironova2019freedom}. The story is one of grassroots transformation pressure accelerated by access to funds: the Free Syrian Army's (FSA) leadership of the Sunni rebel factions began to fracture after 2012, under strain from Islamist groups who could harness growing religiosity and desires for revenge that were transforming the ranks of many militias~\autocite[106]{abboud2018syria}. Thus, fighters dissatisfied with the relative moderation of the FSA command were able to defect and migrate to the new, more extreme, groups~\autocite{abboud2018syria}.  %more about fluidity in conflict zones.

\section{Factors that Increase Accommodation and the Personnel Resource Curse}

Militant groups are both uniquely susceptible to bottom-up pressures and also have specific contextual factors that make their leaders experience accommodation pressures differently than non-militant groups. The following section presents features that should increase the likelihood of observing bottom-up transformation through leader accommodation. 

As described in the previous chapter, survival threats can predispose leaders to experiencing the personnel resource curse. For the most part, militant leaders face inherent existential survival threats. Even groups that are strong enough to reasonably expect ultimate concessions can still lose out if the tide of the conflict turns against them.\footnote{One important exception are groups who are able to credibly expect to imminently participate in a post-conflict political system, either through victory or a negotiated settlement.}

However, even within a general instability and survival threats that can be found in an armed conflict, certain exogenous shocks and features of a conflict's dynamics should increase the likelihood of leaders being pressured to accommodate their base.

\subsection{Mobilizing}

Staffing fledgling clandestine militant groups is one place to observe the effects of an intense survival threat. Prospective terror group leaders must launch their new organizations, which often leads them to develop a two-stage recruitment model. First groups reach for personnel, often by turning to prisons as a ready pool of potential members. As they become stronger and more established, successful terror groups begin to more carefully screen and select prospective members~\autocite{bloom2017constructing}.

Once established, organizations may experience temptation to re-mobilize and reach into a new membership demographic. The new members sought in this form of mobilization can shape the group in much the same way as an initial endowment of members can.  Radical British environmentalists in the second half of the $20^{th}$ century recounted experiencing a similar bottom-up transformation. A 2003 periodical released by, and for, activists in the \say{ecological resistance} community bitterly traced what the editor(s) viewed as erosion of the seriousness of the movement. 

In an op-ed announcing their intention to focus on empowering lone wolf activists, the editor(s) of the tenth issue of an ecoterrorist publication, titled \textit{Do or Die}, complained that changing recruitment strategies to encourage wider social involvement in the movement resulted in a class of members who were \say{attracted to \say{campaign} jobs...[and] inclined to paper pushing rather than physical action}~\autocite[3]{doordie2003rise}.  As well, they blamed attempts to increase the popular appeal of radical
environmentalism with instigating \say{a terrible internal pressure
crushing the radical content and practical usefulness of the
groups}~\autocite[3]{doordie2003rise}. As an example of this process,
the document cited Greenpeace ejecting a director, Paul Watson, for
failing to moderate his activities.  In the editorial's telling, the
need to satisfy these new members resulted in curtailing the
activities of the more extreme wing due to concerns that illegal
actions would alienate new and prospective members.  

\subsection{Recruiting Factions}
Leaders can end up subjected to the personnel resource curse by admitting cohesive social groups that later turn into factions. Strong social ties among the recruits reduce top-down socializing capacity, and thus makes accommodation more likely. 

These ties work through a number of pathways. First, dense interconnections within a community make individuals less likely to assimilate to the  preferences and priories of existing group insiders. Instead, by having a strong existing network, recruits can turn to their own ties and reduce their absorption of the new message~\autocite{morrison2002newcomers}.  Secondly, within militant organizations, strong relational ties can make fighters and commanders more willing to disobey instructions~\autocite{hundman2019rogues}. Third, dense community networks make it more likely that followers will be loyal to a particular faction or commander, rather than the group as a whole. The threat that members will respond to a disagreement by choosing the commander or faction over the leader can motivate leader accommodation, lest the disagreement harden into a factional schism.

Strong factions and internal coalitions can drive organizational changes, often capturing the direction of strategic evolution of a group~\autocite{cyert1963behavioral, march1958organizations, harmel1994integrated,bacharach1980power,pfeffer1981power, bettcher2005factions, gray1985politics}. This can also lead to internal fissures, thereby reducing operational efficacy and increasing the risk of schism and collapse. For example, the stress of factional politics can lead militant organizations to splinter~\autocite{pearlman2012nonstate, bakke2012plague}. Finally, competition between factions can have lethal implications for bystanders in a conflict theater~\autocite{de2008terrorist, bloom2004palestinian}.

\subsection{Community Shocks}
External shocks and sudden military successes can rapidly change the political calculus of actors in a conflict and, in one fell swoop, both mobilize new partisans and make the group more attractive. In turn, leaders may view these potential recruits as way of ensuring short-term survival or capitalizing on battlefield momentum in their favor. 

The Sandinista National Liberation Front (FSLN) in Nicaragua illustrates how external community shocks can initiate bottom-up transformations. As~\textcite{mosinger2017dissident} details, in 1967 and 1972, \say{grievance-triggering focus event[s]} motivated new constituencies to regard the FSLN as a viable avenue through which to express anti-state grievances.  In 1967, the violent repression of a demonstration mobilized radical student organizations. Five years later, in 1972, government mismanagement of relief efforts after the Managua earthquake mobilized Christian activists. Recruits from the new constituencies then flocked to the FSLN and created new internal factions and external bases~\autocite[210]{mosinger2017dissident}. Following both recruitment shocks, the FSLN was riven by internal power struggles as the new members sought to advance their preferences within the group. 

\subsection{Successes}
Success can also bring inflows of recruits who drive transformation. Indeed, the history of militant movements often features military or political gains followed by internal stress as the organization tries to manage the recruits who responded to the battlefield momentum by rushing in. Two illustrative examples demonstrate how a windfall of recruits following successes can seed long-term problems. 

In 1968, the Palestine Liberation Organization claimed credit for fighting the Israeli Army to a stalemate in Karameh, Jordan. Reaping the rewards of a symbolic victory, the movement quickly gained thousands of new Palestinian and Arab recruits~\autocite{sharif2009arafat}. However, this bounty rapidly turned toxic, as the new manpower quickly exceeded the PLO's absorption capacity, and the new fighters began abusing their host population in Jordan~\autocite{szekely2017politics}. This abuse exacerbated tensions between the PLO and their Jordanian and Lebanese hosts, undermining the Executive Committee's strategic goal to remain on good terms with their sponsors~\autocite{szekely2017politics}.

Seven years later, in 1975, a founder of the Eritrean Liberation Front (Jebha), Said Hussein, returned to the group after nine years in prison only to discover his organization transformed.  A nationalist group formerly dominated by conservative highland Muslims, the Jebha militia had been molded by an influx of Christians after Ethiopian crackdowns in 1974 and 1975. Indeed, after one crackdown the number of prospective members so exceeded Jebha's absorption capacity that the group asked potential members to remain home until camp space opened~\autocite[155]{woldemariam2016battlefield}. The new members, largely drawn from lowland Christian communities, quickly began pushing for Jebha to adopt a Marxist ideology anathema to the original founders' socially conservative inclinations~\autocite[111]{woldemariam2018insurgent}. 

\subsection{Foreign Backing}

Organizations receiving foreign backing should also be particularly susceptible to a decision making process that prioritizes rapid expansion even at the expense of cohesion. This occurs on two fronts: first, when backers provide fungible resources, militant organizations are incentives to quickly deploy the resources, lest they become lost to graft, capture, or other inefficiency. Secondly, backers often provide funding to militant groups with the expectation of an increased operational profile. The need to increase activities then drives leaders to rapidly expand their personnel. 

\subsection{New Conflict Actors}

The first is the entry of a new antagonist into the conflict. This enhances the risk of failure that the group faces by not only implying more battles, but also leading to a greater need for resources so as to carry out a multi-front conflict. Leaders should come under greater pressure to accommodate because they have a strong incentive to recruit very quickly to reinforce their own capabilities.  This pressure to rapidly grow can enhance the temptation to overlook lack of alignment in order to gain capabilities. These capabilities can mean fighters and manpower, but also access to communities and local networks. If the latter, the leader may end up accommodating yet again, as the local connections that the new recruits bring also help the new recruit to withstand socialization pressure. A militia commander in Taiz, Yemen succinctly described how survival threats can induce leaders to accept personnel due to short-term considerations, even knowing that those same members are likely to generate long-term problems. In explaining why he allied with al-Qaeda fighters despite not supporting their ideology, the commander explained: \say{when you are days, if not hours from being over-run, you do not care where the supplies or men come from or what their beliefs are so long as they can fight and are fighting the same enemy you are...we can sort out al-Qa'ida after we've beaten the Houthis}~\autocite[7]{horton2017fighting}.

Another way in which change to a conflict environment can spur accommodation pressure on leaders and thus drive bottom-up change is, counter-intuitively, the introduction of new conflict actors on the same side. If these new actors are similar enough that recruits can credibly threaten to defect, recruits and brigades can demand greater accommodation from the leader. In this case, the leader comes under greater pressure to accommodate because exit is less costly for the recruits. %example from my bank of fighter fluidity

\section{Where to Expect Less Bottom-Up Change}

In the context of militant groups, we should expect that leaders with a greater socializing and monitoring ability should be less susceptible to bottom-up transformation. Leaders can shape their socializing ability by seeking to change their physical capabilities as well as by influencing the information environment of their personnel.

Territorial control--- and particularly safe havens--- should dramatically increase a leader's socializing ability. In particular, safe havens allow leaders to establish training camps, which are powerful tools to enhance a leader's socializing capabilities. Uncontested training camps centralize intake and allow the leader to capture the attention and activities of their new members, and thus build an all-encompassing community that permits more compete socialization by reducing competing messages and enhancing vertical ties among their members.

We should also expect less accommodation to grassroots preferences when the leader is able to reduce options for exit. Groups that exemplify this type of behavior include those that forcibly recruit fighters and, particularly, those that force their fighters to commit human rights violations in their home territories~\autocite{beber2010industrial, dudenhoefer2016understanding, eck2014coercion, gates2017membership, peters2011group}.

The Islamic State of Iraq and al-Sham (ISIS) demonstrates how militant groups can take advantage of technological changes to increase their socialization capacity. ISIS is an outlier in adopting and magnifying two trends that reduce the need to accommodate. In creating multifacted online propaganda platforms, ISIS generated a tool to monopolize the attention of prospective recruits even before the recruits arrive in on the ground in their territories. This pushes against the increasing difficulty of establishing all-encompassing training camps in contexts in which camps are military targets and during conflicts in which new recruits are quickly moved into operational capacities.\footnote{Notably, the first American airstrike against ISIS's wing in Yemen was described in press releases as intentionally reducing the group's training and socialization capacities by \say{disrupting the organization's attempts to train new fighters.}~\autocite{dod2017airstrike}} Secondly, both ISIS and their precursor, al-Qaeda in Iraq are reported to have used high-casualty operations to systematically change their internal demographics. In particular, the group appears to systematically deploy their most extreme foreign fighters to carry out commando and suicide operations, thereby gaining an advantageous tactical advantage while also reducing internal pressure from members who are less willing to compromise.\footnote{For the heavy emphasis on foreign fighters as suicide bombings and as commandos, see ~\cite{hafez2007suicide, reuter2015butcher, weiss2015isis}}. An indirect reference to the use of suicide bombings as a tool to manage the internal expectations comes from a member of an ISIS assault team, who described ISIS conducting a needless attack to satisfy internal demand for suicide operations:
\say{Once a group had a suicide mission volunteer detonate a car filled with explosives under a bridge [...] But there was no enemy near the bridge, and we could have just gone there at night, quietly positioned the explosives, and detonated them remotely. There was absolutely no need for a suicide operation}~\autocite[163]{mironova2019freedom}

\section{Implications for International Relations}
%% Tie-in to big-picture IR

Considering how the personnel resource curse affects militant groups provides an opportunity to consider the big picture of what this theory brings to our understanding of international relations and international security.

On the one hand, it contributes to our understanding of the dynamics of opaque organizations. Revolutionary and clandestine movements are particularly susceptible to bottom-up transformations, because they often exist on a razor edge of survival, and want growth, but are also wedded to an ideological perspective. My theory illuminates constraints that shape the decision making of these leaders.

My theory also generates leverage for practitioners of counter terrorism and counter insurgency by providing a systematic way of thinking about the options for creating alternative mobilization outcomes. Applying the personnel resource curse theory produces tools for thinking about when there could be internal fissures and stressors that can be exploited to either decapitate an organization, encourage splintering of a critical base, or— less obviously— suggest that policymakers should accelerate the bottom-up transformation process to tie a transnational group more closely to a specific context.

As well, the theory is particularly salient to considering the dynamics and consequences of proxy and internationalized conflicts. As the international system continues to be characterized by policies of indirect control, providing a theory that can help predict when an agent organization will begin to diverge from what the backers expect is critical. The activity of considering how inversions of internal leverage constrain the options of group leaders can be used to condition expectations of how the militarily-salient bulk of an organization will respond to changes in demands from backers. 
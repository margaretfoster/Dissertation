In this outline of the theory section, more attention to how to integrate the model into the subsequent discussion.

I.  Chapter introduction 

II. Model introduction

III. Equilibrium and comparative statics

IV. Discussion: how do contextual factors influence

1. Socialization
2. Probability of successes linked to recruit resources
Expand on what the recruit might bring into the organization; contextual trade-off between what they bring in and other parameters that influence outcome. 
What could resources be: social and material
    A. When resources are social, this often interacts with the resistance to socialisation term (RR), because the connections that make the recruit(s) valuable also make it less desirable for the organization to monopolize their time and attention, which is one of the more effective methods of socialization. This occurs because when resources are social, increasing socialization capacity decreases the resources that will be needed.
    
    B. Resources are skills: that heightens the effect that outside options has on the leader's need to accommodate. Exit options and desirability of other organizations can be unpacked in the UG term.
    C. If Resources are material. [[idk-- seems like a subcategory 
    of skills in that both are easily fungible, but don't necessarily increase resistance to socializing if